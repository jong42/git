\documentclass[]{scrbook}

\usepackage{tgtermes}
\usepackage[scale=0.92]{tgheros}
\usepackage{tgcursor}

\usepackage[ngerman]{babel}
\usepackage[utf8]{inputenc}
\usepackage{microtype}
\usepackage[numbers]{natbib}
\usepackage{graphicx}
\usepackage{xcolor}
\usepackage{booktabs}
\usepackage{amsmath,amssymb}
\usepackage{amsthm}
\usepackage{amsmath}
\usepackage{thmtools}
\usepackage{tikz}
\usetikzlibrary{3d,calc}
\usepackage{float}
\usepackage{placeins}
\usepackage{caption}


\usepackage{hyperref}
\usepackage{enumitem}

\usepackage{tabularx}
\usepackage{listings}

\usepackage{cancel}

\usepackage[nonumberlist,nopostdot,numberedsection,toc]{glossaries}
\makeglossaries

\declaretheorem[style=definition,qed=$\diamond$,numberwithin=chapter]{Definition}
\declaretheorem[style=definition,qed=$\diamond$,sibling=Definition]{Beispiel}
\declaretheorem[style=definition,qed=$\diamond$,sibling=Definition]{Notation}
\declaretheorem[style=plain,qed=$\diamond$,sibling=Definition]{Satz}
\declaretheorem[style=plain,qed=$\diamond$,sibling=Definition]{Lemma}
\declaretheorem[style=plain,qed=$\diamond$,sibling=Definition]{Korollar}
\declaretheorem[style=plain,qed=$\diamond$,sibling=Definition]{Theorem}
\declaretheorem[style=remark,qed=$\diamond$,sibling=Definition]{Bemerkung}

\newcommand{\R}{\mathbb R}
\newcommand{\tr}{^\mathsf{T}}
\DeclareMathOperator{\Id}{Id}
\DeclareMathOperator{\diag}{diag}
\DeclareMathOperator{\epi}{epi}
\DeclareMathOperator{\co}{co}
\DeclareMathOperator{\interior}{int}
\DeclareMathOperator{\Proj}{Pr}

\newcommand{\midd}{\mathrel{}\middle|\mathrel{}}
\newcommand{\F}{\mathcal{F}}
\newcommand{\norm}[1]{\left\lVert#1\right\rVert}
\newcommand{\N}{\mathcal{N}}

\begin{document}

\title{Konvexe Optimierung}
\date{\today}
\maketitle
\tableofcontents
\chapter{Einf\"uhrung in die Optimierung}

\section{Grundlagen}

\begin{Definition}
	\begin{center}
	Ein (mathematisches) \textbf {Optimierungsproblem}\index{allgemeines Optimierungssystem} hat die folgende Form\\
	\begin{gather}
  		\label{eq:P}   
  		\tag{P}
  			\begin{aligned}
    			\min_x
    			& & & f(x) \\
    			\text{s.t.}
    			& & & x\in \mathcal{F}
  			\end{aligned}
	\end{gather}
	\begin{itemize}
		\item  	$x=(x_1,\cdots,x_n)\tr\in\R^n$ sind die \textbf{Variablen}
		\item	$f\colon D\to\R, D\subset\R^n$, ist die \textbf{Zielfunktion}
		\item	$\mathcal{F}$ ist die \textbf{zul\"assige Menge}; ihre Elemente hei\ss en \textbf{zul\"assige Punkte}\qedhere
\end{itemize}
\end{center}
\end{Definition}

\paragraph{Die Nebenbedingungen}


\begin{itemize}
\item  Ist $\mathcal{F}=D$, dann spricht man von einem \textbf{unrestringierten Problem}.
\item Wird $\mathcal{F}$ durch Nebenbedingungen (Restiktionen) definiert, dann hei\ss t \eqref{eq:P} \textbf{restringiertes Optimierungsproblem} oder \textbf{Optimierungsproblem mit Nebenbedingungen}; $\mathcal{F}$ ist hierbei typischerweise durch Gleichungen und Ungleichungen definiert:
\begin{equation*}
	\mathcal{F}=\{x\in\R^n \mid h(x)=0,~ g(x)\leq 0\}
\end{equation*}

\paragraph{Einfache Beispiele}


\begin{enumerate}[label=\emph{\alph*})]
\item Mit $\mathcal{F}=D=\R$ sei $f\colon \R\to\R$ durch $f(x)=x^2$ definiert. Dann ist $x^\ast = 0$ die eindeutig bestimmte L\"osung von \eqref{eq:P}.
\item Mit $\mathcal{F}=D=\R$ sei $f\colon \R\to\R$ durch $f(x)=\sin(x)$ definiert. Dann hat \eqref{eq:P} unendlich viele L\"osungen $x^\ast=2k \pi-\frac{\pi}{2} ,k\in \mathbb{Z}$.
\item Mit $\mathcal{F}=D=\R$ sei $f\colon \R\to\R$ durch $f(x)=x$ definiert. In diesem Fall ist $f$ auf $\mathcal{F}$ nicht nach unten beschr\"ankt und \eqref{eq:P} hat keine L\"osung.
\item Mit $\mathcal{F}=D=\R$ sei $f\colon \R\to\R$ durch
\begin{equation*}
	f(x)= (2x-2)^2(3x+3)^2+10x
\end{equation*}
definiert. Dann ist der Punkt $x^\ast = -1$ globale L\"osung von \eqref{eq:P}. Der Punkt $\tilde{x}=1$ ist eine lokale L\"osung.
\item Wir betrachten das Problem
	\begin{gather}
	\label{eq:P2}   
	\tag{P2}
	\begin{aligned}
	\min_x
	& & & f(x) = x^3\\
	\text{s.t.}
	& & & x\geq 1
	\end{aligned}
	\end{gather}
Hier ist $D=\R$ und $\mathcal{F}=\{x\in\R \mid x\geq1 \}$. Der Punkt $x^\ast = 1$ ist L\"osung. Ohne die Restriktion h\"atte das Problem \eqref{eq:P2} keine L\"osung.
\end{enumerate}
\end{itemize}

\begin{Definition}(lokaler Minimalpunkt)~ \\
	Ein Punkt $x^\ast\in\F$ hei\ss t \textbf{lokaler Minimalpunkt} von $f$ auf $\F$ oder \textbf{lokale L\"osung} von \eqref{eq:P}, falls es ein $r>0$ mit
	\begin{equation*}
		f(x)\geq f(x^\ast) \qquad \forall x\in\F \cap B(x^\ast,r)
	\end{equation*}
	gibt. \vspace{5pt}\\
	Ein Punkt $x^\ast\in\F$ hei\ss t \textbf{strikter lokaler Minimalpunkt} von $f$ auf $\F$ oder \textbf{strikte lokale L\"osung} von \eqref{eq:P}, falls es ein $r>0$ mit
	\begin{equation*}
	f(x)> f(x^\ast) \qquad \forall x\in\F \cap B(x^\ast,r),\; x\neq x^\ast
	\end{equation*}
	gibt.
\end{Definition}

\begin{Bemerkung} Eine offene Kugel mit Radius $r$ um $x$ ist definiert als,
\begin{equation*}
	B(x,r) = \lbrace y\in\R^n ~\mid \norm{y-x}<r \rbrace.
\end{equation*}
\begin{center}
	\begin{tikzpicture}
	\shade[ball color = gray!40, opacity = 0.4] (0,0) circle (2cm);
	\draw (0,0) circle (2cm);
	\draw (-2,0) arc (180:360:2 and 0.6);
	\draw[dashed] (2,0) arc (0:180:2 and 0.6);
	\fill[fill=black] (0,0) circle (1pt);
	\draw[dashed] (0,0 ) -- node[above]{$r$} (2,0);
	\end{tikzpicture}
	\captionof{figure}{Beispiel f\"ur $B(x,r)$ f\"ur $x\in\R^3$}
\end{center}
\end{Bemerkung}


\begin{Definition}(globaler Minimalpunkt)~ \\
	Ein Punkt $x^\ast\in\F$ hei\ss t \textbf{globaler Minimalpunkt} von $f$ auf $\F$ oder \textbf{globale L\"osung} von \eqref{eq:P}, falls 
	\begin{equation*}
	f(x)\geq f(x^\ast) \qquad \forall x\in\F 
	\end{equation*}
	gibt. \vspace{5pt}\\
	Ein Punkt $x^\ast\in\F$ hei\ss t \textbf{strikter globaler Minimalpunkt} von $f$ auf $\F$ oder \textbf{strikte globale L\"osung} von \eqref{eq:P}, falls 
	\begin{equation*}
	f(x)> f(x^\ast) \qquad \forall x\in\F ,\; x\neq x^\ast
	\end{equation*}
	gibt.
\end{Definition}

\paragraph{Einfache Beispiele}

\begin{enumerate}[label=\emph{\alph*})]
	\setcounter{enumi}{5}
	\item Die Funktion $f(x)=\sin(x)$ hat unendlich viele globale Minima.
	\item  Die Funktion
		\begin{equation*}
			f(x_1,x_2)=\frac{1}{2}(x_1^2+x_2^2)-\cos(x_1^2)-\cos(x_2^2)
		\end{equation*}
		hat ein striktes globales Minimum und noch weitere strikte lokale Minima. Die Funktion besteht aus einem quadratischen Term und noch zwei weiteren Termen, die man als \glqq Rauschen\grqq ~ interpretieren kann.
\end{enumerate}

\paragraph{Maximierung}

\begin{itemize}
	\item Oft soll die Zielfunktion $f$ maximiert werden, d.\,h., wir suchen ein $x^\ast \in \F$ mit
	\begin{equation*}
		f(x)\leq f(x^\ast) \qquad \forall x\in\F
	\end{equation*}
	\item Diese Aufgabe ist \"aquivalent dazu, ein $x^\ast\in\F$ zu finden mit
	\begin{equation*}
		-f(x)\geq -f(x^\ast)\qquad \forall x\in\F
	\end{equation*}
	\item[$\Rightarrow$] Äquivalentes Problem:
	\begin{gather}
	\label{eq:PMax}   
	\tag{PMax}
	\begin{aligned}
	\min 
	& & & g(x)=-f(x)\\
	\text{s.t.}
	& & & x\in\F
	\end{aligned}
	\end{gather}
	\item  Es gen\"ugt daher, nur Minimierungsaufgaben zu betrachten.
\end{itemize}
	
\section{Beispiele}
	\paragraph{Portfolio-Optimierung}
	\begin{itemize}
		\item Variablen: Aufteilung des Portfolios
		\item Nebenbedingungen: Budget, Beschr\"ankungen bei Investitionen
		\item Zielfunktion: maximale Rendite, minimales Risiko
	\end{itemize}
	\paragraph{Optimale Steuerung eines Raketenautos}
	\begin{itemize}
		\item Variablen: Beschleunigung des Autos
		\item Nebenbedingungen: Dynamik, maximale Beschleunigung
		\item Zielfunktion: Zielort erreichen, Verbrauch an Treibstoff minimieren
	\end{itemize}
	\paragraph{Data Fitting}
	\begin{itemize}
		\item Variablen: Modell-Parameter
		\item Nebenbedingungen: geg.\ Informationen, Parameterbeschr\"ankungen
		\item Zielfunktion: Abweichung/Fehler minimieren
	\end{itemize}
	
\subsection{Portfolio-Optimierung}
\begin{itemize}
	\item Es werden $n$ Wertpapiere gehandelt, wobei $R_j$ die Rendite des $j-$ten Wertpapiers in der n\"achsten Zeitperiode darstellt (\textbf{Zufallsvariable}).
	\item Ein \textbf{Portfolio} besteht aus einer Zusammenstellung dieser Wertpapiere, dargestellt durch nichtnegative Zahlen $x_j\in\R^+,j=1,\dots,n$.
	\item Rendite eines gegebenen Portfolios und die erwartete Rendite:
	\begin{equation*}
		R=\sum\limits_{j=1}^n x_jRj \qquad \mathbb{E}[R]=\sum\limits_{j=1}^n x_j\mathbb{E}[R_j]
	\end{equation*}
	\item Als Ma\ss{}  f\"ur das Risiko nutzen wir die durchschnittliche absolute Abweichung vom Erwartungswert
	\begin{equation*}
		\mathbb{E}[R-\mathbb{E}[R]]=\mathbb{E}\Bigg\lvert\sum\limits_{j=1}^n x_j (R_j-\mathbb{E}[R_j])\Bigg\rvert
	\end{equation*}
\end{itemize}
Es ergibt sich folgendes Optimierungsproblem:
\begin{eqnarray*}
		\max_{x_1,\dots,x_n} & &\mu\sum\limits_{j=1}^nx_j \mathbb{E}[R_j] - \mathbb{E}\Bigg\lvert\sum\limits_{j=1}^n x_j (R_j-\mathbb{E}[R_j])\Bigg\rvert\\
		\text{s.t.} & & \sum\limits_{j=1}^n x_j = 1\\
		& & x_j \geq 0, \qquad j=1,2,\dots,n
\end{eqnarray*}

\begin{itemize}
	\item Die beiden gegens\"atzlichen Ziele (erwartete Rendite maximieren vs. Risiko minimieren) werden durch den Parameter $\mu$ gewichtet
	\item Approximieren wir $\mathbb{E}[R_j]$ durch den Mittelwert der letzten beobachteten Renditen, l\"asst sich dieses Problem in ein \textbf{lineares Optimierungsproblem} transformieren
\end{itemize}

\subsection{Optimale Steuerung - Raketenauto}
Wir betrachten ein Auto mit Raketenantrieb, dass nur geradeaus f\"ahrt. Durch den Schub der Rakete kann das Auto sowohl nach links als auch nach rechts beschleunigen. Vom Startpunkt $x_0$ aus soll ein Ziel $x_f$ m\"oglichst genau in gegebener Zeit $t_f$ erreicht werden.
  
\begin{center}
    \begin{tikzpicture}[scale=0.9]
       \useasboundingbox (-3,-0.8) rectangle (9,1.3);
       % x-Achse
       \draw[very thick] (-3,0) -- (4,0);
       \draw[very thick,dotted] (4.1,0) -- (5.0,0);
       \draw[very thick,->] (5.1,0) -- (9,0);
       \draw[thick] (0,0) -- (0,-0.12); \draw (0,-0.35) node {$x_0$};
       \draw[thick] (8.5,0) -- (8.5,-0.12); \draw (8.5,-0.35) node {$x_f$};
       % Duesenauto, Karosserie
       \fill[red] (-2.2,0.1) -- (0,0.1) -- (-0.7,0.5) -- (-2,0.5);
       % Duesenauto, Raeder
       \draw[fill=black!50] (-0.7,0.2) circle (0.2);
       \draw[fill=black!90] (-0.7,0.2) circle (0.1);
       \draw[fill=black!50] (-1.7,0.25) circle (0.25);
       \draw[fill=black!90] (-1.7,0.25) circle (0.15);
       % Duesenauto, Duese
       \fill[red] (-1.2,0.5) -- (-1.2,0.6) -- (-1.3,0.6) -- (-1.3,0.5);
       \fill[red] (-1.8,0.5) -- (-1.8,0.6) -- (-1.9,0.6) -- (-1.9,0.5);
       \fill[black] (-2,0.6) -- (-1.1,0.6) -- (-1.1,0.9) -- (-2,0.9);
       % Duesenauto, Duesenstrahl
 	  \fill[left color=orange, right color=orange!20!blue] (-2,0.78) -- (-2,0.73) -- 
         (-2.8,0.63) -- (-2.8,0.70) -- (-2.85,0.755) --(-2.8,0.81) -- (-2.8,0.88) -- (-2,0.78);
       %
       \draw[very thick,->,dashed] (0,0.3) -- (5.8,0.3);
       %
       % Duesenauto, Karosserie
       \fill[red] (5.8,0.1) -- (8,0.1) -- (7.3,0.5) -- (6,0.5);
       % Duesenauto, Raeder
       \draw[fill=black!50] (7.3,0.2) circle (0.2);
       \draw[fill=black!90] (7.3,0.2) circle (0.1);
       \draw[fill=black!50] (6.3,0.25) circle (0.25);
       \draw[fill=black!90] (6.3,0.25) circle (0.15);
       % Duesenauto, Duese
       \fill[red] (6.8,0.5) -- (6.8,0.6) -- (6.7,0.6) -- (6.7,0.5);
       \fill[red] (6.2,0.5) -- (6.2,0.6) -- (6.1,0.6) -- (6.1,0.5);
       \fill[black] (6,0.6) -- (6.9,0.6) -- (6.9,0.9) -- (6,0.9);
       % Duesenauto, Duesenstrahl
 	  \fill[left color=blue, right color=blue!20!orange] (6.9,0.78) -- (6.9,0.73) -- 
         (7.7,0.63) -- (7.7,0.70) -- (7.75,0.755) --(7.7,0.81) -- (7.7,0.88) -- (6.9,0.78);
    \end{tikzpicture}
    \captionof{figure}{Düsenauto}
\end{center}
Starten wir am Punkt $(x_1(0),x_2(0)=(a_1,a_2)$ und wollen den Punkt $(0,0)$ so gut wie m\"oglich erreichen und zus\"atzlich den Energieverbrauch minimal halten, dann ergibt sich folgendes Optimierungsproblem:
\begin{eqnarray*}
		\min_{x_1,x_2,u} & & x_1(t_f)^2+x_2(t_f)^2+\alpha\int_{0}^{t_f}u(t)^2dt\\
		\text{s.t.} & & \dot{x_1}(t) = x_2(t)  \quad \dot{x_2}(t)=u(t),\\
		& & x_1(0)=a_1, \quad x_2(0)=a_2,\\
		& & b_l\leq u(t) \leq b_u \quad \text{f\"ur fast alle }t\in\left[0,t_f\right].
\end{eqnarray*}
Der Regularisierungsparameter $\alpha$ gewichtet hier zum einen die beiden Optimierungsziele, zum anderen gl\"attet er die optimale Steuerung. Gel\"ost wird diese Optimierungsproblem numerisch, indem man die Steuerung st\"uckweise konstant approximiert und die gew\"ohnlichen Differentialgleichungen der Nebenbedingungen mit dem Euler-Verfahren diskretisiert. Man erh\"alt hierbei ein \textbf{linear-quadratisches Optimierungsproblem}.
	
\begin{center}
     \begin{tikzpicture}[scale=0.88]
       \draw[draw=black!10,line width=0.25] (0,-1.5) grid[step=0.5] (5,1.5);
       \draw[draw=black!20,line width=0.5] (0,-1) grid[step=0.25] (5,1);
       \draw[thick,->] (0,-1.5) -- (5,-1.5) node[below] {\footnotesize{$t$}};
       \draw[thick,->] (0,-1.5) -- (0,1.5) node[left] {\footnotesize{$u$}};
       \foreach \u in {-1,0,1} {
         \draw (0,\u) node[left] {\footnotesize{$\u$}};
         \draw (-0.1,\u) -- (0,\u);
       }
       % Steuerung u
       \draw[ultra thick,blue] plot[domain=0:3.5174292] (\x,-1);
       \draw[ultra thick,blue] plot[domain=3.5174292:5] (\x,1);
       \draw[dashed] plot coordinates {(3.5174292,1.5) (3.5174292,-1.5)};
     \end{tikzpicture}
     %
     \begin{tikzpicture}[scale=0.88]
       \draw[draw=black!10,line width=0.25] (0,-1.5) grid[step=0.5] (5,1.5);
       \draw[draw=black!20,line width=0.5] (0,-1) grid[step=0.25] (5,1);
       \draw[thick,->] (0,-1.5) -- (5,-1.5) node[below] {\footnotesize{$t$}};
       \draw[thick,->] (0,-1.5) -- (0,1.5) node[left] {\footnotesize{$u$}};
       \foreach \u in {-1,0,1} {
         \draw (0,\u) node[left] {\footnotesize{$\u$}};
         \draw (-0.1,\u) -- (0,\u);
       }
       % Steuerung u
       \draw[blue,ultra thick] plot file {data/datenualpha.txt};
       \draw[dashed] plot coordinates {(2.4700,1.5) (2.4700,-1.5)};
       \draw[dashed] plot coordinates {(4.7500,1.5) (4.7500,-1.5)};
     \end{tikzpicture}
     \captionof{figure}{Optimale Steuerung f\"ur $\alpha = 0$ (links) und $\alpha = 1$ (rechts)}
\end{center}

\subsection{Lineare Regression}
Gegeben seien Messwerte $(\xi_i,\eta_i), i=1,\dots,m$. Der funktionale Zusammenhang zwischen den $\xi-$ und den $\eta-$ Werten soll durch eine Gerade
\begin{equation*}
	\eta(\xi)=g(\xi;x_1,x_2)= x_1\xi+x_2
\end{equation*}
beschrieben werden. Aufgrund von Messfehlern liegen nicht alle Messwerte auf der Geraden. Ziel ist es den Parameter $x=(x_1,x_2)\tr\in\R^2$ so zu bestimmen, dass die zugeh\"orige Gerade {\glqq optimal\grqq} zu den Messwerten passt.
   \begin{center}
     \begin{tikzpicture}
       \draw[->,thick] (0,0) -- (7,0) node[below]{$\xi$};
       \draw[->,thick] (1,0) -- (0,0) -- (0,5) node[left]{$\eta$};
       \draw (1,1) node{$\circ$} node[below right]{\footnotesize{$(\xi_1, \eta_1)$}};
       \draw (2.5,2) node{$\circ$} node[above left]{\footnotesize{$(\xi_2, \eta_2)$}};
       \draw (4,2.4) node{$\circ$} node[below right]{\footnotesize{$(\xi_3, \eta_3)$}};
       \draw (6,3.7) node{$\circ$} node[above left]{\footnotesize{$(\xi_4, \eta_4)$}};
       \draw[thick,red] (0,0.7) -- (7,4) node[below right]{$g(\xi; x_1, x_2)$};
     \end{tikzpicture}
     \captionof{figure}{Lineare Regression}
   \end{center}
Das \"ubliche Kriterium f\"ur Optimalit\"at ist die Minimierung der Summe der Fehlerquadrate in den Messpunkten. Dazu definieren wir die Zielfunktion $f: \R^2 \to \R$ durch
\begin{equation*}
	f(x)= \sum\limits_{i=1}^m\left[g(\xi_i,x) - \eta_i \right]^2 = \sum\limits_{i=1}^m\left[x_1\xi_i+x_2-\eta_i \right]^2
\end{equation*}
Das resultierende Optimierungsproblem
\begin{equation*}
	\min_{x_1,x_2}\sum\limits_{i=1}^m\left[x_1\xi_i+x_2-\eta_i \right]^2
\end{equation*}
ist \textbf{unrestringiert} und die Zielfunktion ist \textbf{quadratisch}.
\newpage

%%%%%%%%%%%%%%%%%%%%%%%%%%%%%%%%%%%%%%%%%%%%%
\chapter{Konvexe Optimierungsprobleme}
%%%%%%%%%%%%%%%%%%%%%%%%%%%%%%%%%%%%%%%%%%%%%

\begin{itemize}
\item Historisch standen lineare Probleme im Fokus der Optimierung.
\item Urspr\"ungliche Unterscheidung in lineare und nichtlineare Probleme.
\item  Aber: Bestimmte nichtlineare Probleme k\"onnen effizient gel\"ost werden.
\item  Daher unterscheidet man zwischen \textbf{konvexen und nichtkonvexen Problemen}.
\end{itemize}
Das auf Seite \pageref{eq:P} vorgestellte Problem \eqref{eq:P}
\begin{gather}
\tag{P}
\begin{aligned}
\min_x
& & & f(x) \\
\text{s.t.}
& & & x\in \mathcal{F}
\end{aligned}
\end{gather}
ist genau dann ein \textbf{konvexes Optimierungsproblem}, wenn die zul\"assige Menge $\F$ konvex ist und die Zielfunktion $f$ konvex auf $\F$ ist.

\paragraph{Probleme mit Gleichungs- und Ungleichungsrestriktionen}~\\

Typischerweise ist die zul\"assige Menge durch Gleichungen und Ungleichungen definiert.
Das Problem
 \begin{gather}
   \label{eq:QP}   
   \tag{QP}
   \begin{aligned}
     \min_x
     & & & f(x) \\
     \text{s.t.}
     & & & h_i(x) = 0\,, \quad i = 1,\ldots,m\,,\\
     & & & g_j(x) \leq 0\,, \quad j = 1,\ldots,p\,.
   \end{aligned}
 \end{gather}
ist konvex, wenn $f$ und $g_j,j=1,\dots,p$ konvexe Funktionen sind, und die Funktionen $h_i,i=1,\dots,m$ affin-linear sind, d.h.,
\begin{equation*}
	h_i(x)=a_i\tr x+b_i,\qquad i=1,\dots,m,
\end{equation*}
mit $a_i\in\R^n$ und $b_i\in\R$.

\section{Konvexe Mengen}

\begin{Definition}(konvexe Menge)\\
	Eine Menge $C\subseteq\R^n$ hei\ss t konvex, falls f\"ur beliebige $x,y\in C$ gilt
	\begin{equation*}
	\lbrace (1-t)x+ty ~\vert~ 0\leq t \leq 1\rbrace  \subseteq C
	\end{equation*}
	d.h., f\"ur die Verbindungsstrecke $\left[x,y\right]=\{(1-t)x+ty ~\vert~ 0\leq t \leq 1 \}$ gilt $\left[x,y\right]\subseteq C$.
\end{Definition}
\begin{center}
	\begin{tikzpicture}
	\draw[rotate=-45,fill=gray!25] (0,0) ellipse (30pt and 45pt);
	\textcolor{blue}{
		\draw (-0.5,-0.5) -- (0.7,0.7);
		\fill (-0.5,-0.5) circle[radius=1.5pt];
		\fill (0.7,0.7) circle[radius=1.5pt];
	}
	%\draw (-0.7,0.6) node[below] {\Large $C$};
	\end{tikzpicture}
	\qquad \qquad
	\begin{tikzpicture}
	\useasboundingbox (-1,-1.35) rectangle (1.5,1.35);
	\draw[fill=gray!25] (0,0) to [out=140,in=90] (-1,-1)
	to [out=-90,in=240] (0.8,-0.6)
	to [out=60,in=-60] (1.2,1.2)
	to [out=120,in=90] (0.3,0.7)
	to [out=-90,in=20] (0.3,0)
	to [out=200,in=-40] (0,0);
	\textcolor{red}{
		\draw (-0.5,-0.5) -- (0.7,0.7);
		\fill (-0.5,-0.5) circle[radius=1.5pt];
		\fill (0.7,0.7) circle[radius=1.5pt];
	}
	%\draw (0.3,-0.2) node[below] {\Large $N$};
	\end{tikzpicture}
	\captionof{figure}{links: konvexe Menge, rechts: keine konvexe Menge}
\end{center}


%\begin{Satz}
%F\"ur eine konvexe Optimierungsaufgabe ist jede lokale L\"osung eine globale L\"osung, und die L\"osungsmenge von \eqref{eq:QP}
%\begin{equation*}
%	\mathcal{S}=\{ x\in\F ~\vert~ f(x)\leq f(y~\forall y\in\F)\}
%\end{equation*}
%ist konvex.
%\end{Satz}
%\begin{proof}
%Sp\"ater in der Vorlesung
%\end{proof}
%\begin{Bemerkung}
%	Über die \textbf{Existenz einer L\"osung} haben wir hier noch keine Aussage getroffen!
%\end{Bemerkung}


\subsection{Beispiele}
\begin{enumerate}[label=\emph{\alph*})]
	\item Die konvexen Teilmengen des $\R$ sind, neben $\R$ selbst, die Intervalle. Beispielsweise $\left[a,b \right]$ mit $a,b\in\R,a\leq b,$ oder $\left]a,\infty\right[$ mit $a\in\R$.
	\begin{proof}
	$C=\left[a,b \right]$ mit $a,b\in\R,~a\leq b \Rightarrow C$ ist konvex.\\
	Seien $x,y ~\text{beliebige Punkte aus } C \text{ und } t\in\left[0,1 \right]$. Dann gilt einerseits die Absch\"atzung nach unten
	\begin{equation*}
		(1-t)x+ty \stackrel{a \leq x,y}{\geq} (1-t)a+ta = a
	\end{equation*}
	 und andererseits die Absch\"atzung nach oben durch 
	 \begin{equation*}
		(1-t)x+ty \stackrel{x,y\leq b}{\leq} (1-t)b-tb=b
	 \end{equation*}
	 Das hei\ss t zu zwei beliebigen Punkten $x,y$ aus dem Intervall $\left[a,b \right]$ liegt das davon erzeugte Teilintervall $\left[x,y\right]\subseteq \left[a,b\right]$.
	\end{proof}
	\item F\"ur $x\in\R^n$ und $r>0$ bezeichnen wir mit
	\begin{equation*}
		B(x,r)=\lbrace y\in\R^n ~\vert~ \norm{y-x} <r\rbrace
	\end{equation*}
	die offene Kugel und mit
		\begin{equation*}
	\bar{B}(x,r)=\lbrace y\in\R^n ~\vert~ \norm{y-x} \leq r\rbrace
	\end{equation*}
	die abgeschlossene Kugel mit Radius $r$ um den Punkt $x$. Beide Mengen sind konvex.\footnote{siehe Übungsblatt 1}
	\item Hyperebenen
	\begin{equation*}
		\mathcal{H}(a,b)=\lbrace x\in\R^n ~\vert~ a\tr x=b \rbrace
	\end{equation*}
	mit $a\in\R^n,b\in \R$ und die durch sie definierten Halbr\"aume $\lbrace x\in\R ~\vert~ a\tr x\leq b \rbrace$ sind konvexe Mengen.
	\item Polyeder
	\begin{equation*}
	P(A,b)=\lbrace x\in\R^n ~\vert~ Ax\leq b \rbrace
	\end{equation*}
	mit $A\in\R^{m\times n}$ und $b\in\R^m$ sind konvex.
	\begin{proof}
	Seien $x_1,x_2\in P(A,b)$ und $t\in\left[0,1\right]$ $\Rightarrow Ax_1\leq b \wedge Ax_2\leq b$.\\ Dann gilt
	\begin{eqnarray*}
		A\left[(1-t)x_1+tx_2 \right] & = & A(1-t)x_1+A(tx_2)\\
									& = & (1-t)\underbrace{Ax_1}_{\geq b}+t\underbrace{Ax_2}_{\leq b}\\
									& \leq & (1-t)b + tb =b\\
									& \Leftrightarrow & (1-t)x_1+tx_2 \in P(A,b)
	\end{eqnarray*}
	\end{proof}
	\begin{Bemerkung}
		Ein typischer Fall einer zul\"assigen Menge beschrieben durch Gleichungen und Ungleichungen, ist gegeben durch
		\begin{equation*}
			\bar{P}(A,b,G,r) = \lbrace x\in \R^n ~\vert~ Ax\leq b~,~Gx=r\rbrace
		\end{equation*}
		Wie kann man zeigen, dass diese Menge konvex ist?
		Es gilt:
		\begin{equation*}
			Gx=r \Leftrightarrow \left( Gx\leq r \right) \wedge ( \underbrace{Gx\geq r}_{-Gx\leq -r})
		\end{equation*}
		Damit ist
		 \begin{equation*}
			\Rightarrow \begin{bmatrix}&A\\&G\\-&G\end{bmatrix}x\leq \begin{bmatrix}&b\\&r\\-&r
			\end{bmatrix}
		 \end{equation*} 
		 ein Spezialfall von oben.
	\end{Bemerkung}
\end{enumerate}
\subsection{Schnitt, Skalierung und Verschiebung}
\begin{Lemma}Ist $\left(C_j \right)_{j\in J}$ eine Familie konvexer Mengen mit einer beliebigen Indexmenge $J$, dann ist auch $C=\bigcap\limits_{j\in J} C_j$ konvex.
\end{Lemma}
\begin{proof}
	Folgt unmittelbar aus der Definition.
\end{proof}

\begin{Lemma}
	Ist $C\subseteq \R^n$ konvex, dann ist auch
	\begin{equation*}
		aC + b = \lbrace ax+b ~\vert~ x\in C \rbrace
	\end{equation*}
	konvex f\"ur alle $a\in\R$ und $b\in\R^n$.
\end{Lemma}

\begin{Definition}(Konvexkombination)\\
	Sind $x^{(1)},\dots,x^{(k)}\in\R^n$ und $\alpha_1,\dots\alpha_k \in\R$, dann hei\ss t ein Vektor
	\begin{equation*}
		\sum\limits_{i=1}^k\alpha_ix^{(i)}~ \text{mit} ~ \sum\limits_{i=1}^k \alpha_i = 1~ \text{und} ~ \alpha_i\geq 0, i=1,\dots,k,
	\end{equation*}
	Konvexkombination der Vektoren $x^{(1)},\dots,x^{(k)}$.
\end{Definition}
\begin{Satz}
Eine Menge $C\subset\R^n$ ist genau dann konvex, wenn sie alle Konvexkombinationen von Punkten in $C$ enth\"alt.
\end{Satz}
\begin{proof} siehe Übungsblatt 1
\end{proof}

\begin{Definition}(Konvexe H\"ulle)\\
	F\"ur $A\subset\R^n$ ist die konvexe H\"ulle von $A$, bezeichnet mit $\co A$, die kleinste konvexe Menge, die $A$ umfasst, d.h.
	\begin{equation*}
		\co A = \bigcap\lbrace C\subseteq \R^n ~\vert~ C~\text{konvex}, C\supseteq A \rbrace
	\end{equation*}
\end{Definition}
\begin{Lemma}
	F\"ur $A\subset\R^n$ ist
	\begin{equation*}
		\co A = \lbrace x\in \R^n ~\vert~ x ~\text{ist Konvexkombination von Punkten in }A\rbrace.
	\end{equation*}
\end{Lemma}
\begin{proof}~\\
	Sei $B=\lbrace x\in\R^n ~\vert~ x \text{ ist Konvexkombination von Punkten in } A \rbrace \underbrace{\Rightarrow}_{\text{z.z.}} \co A = B$\\
	$B$ konvex und $A\subseteq B \Rightarrow \co A \subseteq \co B = B$.\\
	Umgekehrt gilt $\forall C$ konvex mit $C\supseteq A \Rightarrow C\supseteq B \underbrace{\Rightarrow}_{C=\co A} \co A \supseteq B$
\end{proof}
\begin{Definition}(Kegel)~\\
	Eine nichtleere Menge $K\subseteq\R^n$ hei\ss t \textbf{Kegel}, wenn mit $x\in K$ auch $tx\in K$ f\"ur alle $t>0$ gilt, d.h., wenn mit $x\in K$ auch der offene Halbstrahl $\lbrace tx ~\vert~ t>0\rbrace\subseteq K$ ist.
\end{Definition}

\begin{Lemma}
	Ein Kegel $K\subseteq \R^n$ ist genau dann konvex, wenn $K+K\subseteq K$ ist.
\end{Lemma}
\begin{proof}
	s. Übungsblatt 1
\end{proof}

\paragraph{Beispiele f\"ur Kegel}

\begin{enumerate}[label=\emph{\alph*})]
	\item Der Kegel $\lbrace x\in\R^n ~\vert~ x \geq 0_n \rbrace$ ist konvex. Es gilt $K+K=K$.
	\item Die Menge
	\begin{equation*}
		\lbrace x\in\R^n~\vert~x\geq 0_n \rbrace\cup\lbrace x\in\R^n~\vert~x\leq 0_n \rbrace
	\end{equation*}
	ist ein Kegel, aber nicht konvex, denn es gilt $K+K= \R^n\not\subseteq K$.
	\item Der Kegel der positiv semidefiniten Matrizen
	\begin{equation*}
		\mathcal{S}^n_+=\lbrace X\in\mathcal{S} ~\vert~ X\succcurlyeq 0\rbrace
	\end{equation*}
	ist konvex, wobei $\mathcal{S}^n$ die Menge der symmetrischen $n \times n$-Matrizen ist.
\end{enumerate}

\paragraph{Wichtige Kegel}

\begin{Definition}
 Ist $S\subseteq \R^n$ und $x\in S$, dann hei\ss t die Menge
 \begin{equation*}
 	K(S,x)=\lbrace t(s-x) ~\vert~ s\in S, ~t>0  \rbrace
 \end{equation*}
 der von $S-x$ ~\textbf{erzeugte Kegel} oder \textbf{konische H\"ulle} von $S-x$.
\end{Definition}
Die Menge $K(S,x)$ besteht aus Halbstrahlen, die durch die Vektoren $S-x$ erzeugt werden, s.d. die Menge nach Definition ein Kegel ist. Wegen $x\in S$ ist immer  auch $0_n\in K(S,x)$ (f\"ur $s=x$) und es gilt $K(S,x)=K(s-x,0_n)$.

\begin{Lemma}
	Sei $C\subseteq \R^n$ konvex und $x\in C$. Dann ist $K(C,x)$ konvex.
\end{Lemma}
\begin{proof}
	cf. undergraduate convexity
\end{proof}
\begin{Definition}
	Ist $C\subseteq\R^n$ konvex und $x\in C$. Dann hei\ss t $s\in\R^n$ \textbf{Normalenrichtung} von $C$ in $x$, wenn 
	\begin{equation*}
		\langle s,y-x \rangle \leq 0 \qquad \forall y\in C
	\end{equation*}
	gilt. Die Menge
	\begin{equation*}
		N(C,x) = \lbrace s\in R^n ~\vert~ s \, \text{ist Normalenrichtung von $C$ in $x$} \rbrace
	\end{equation*}
	hei\ss t \textbf{Normalenkegel} von $C$ in $x$.
\end{Definition}
\begin{Bemerkung}
	Der Normalenkegel ist konvex und abgeschlossen.
\end{Bemerkung}
Ein Beispiel f"ur einen Normalenkegel l"asst sich wie folgt konstruieren.
Seien $C$ ein Unterraum und $x \in C$ beliebig.
Dann ist die konische H"ulle $K(C,x) = C$ und der Normalenkegel $N(C,x) = C^\bot$, also das orthogonale Komplement von $C$.
Der Beweis der Aussage ist eine gute "Ubung.
%
\section{Konvexe Funktionen}

\begin{Definition}(konvexe Funktion)\\
	Ist $D\subseteq \R^n$ und ist $\emptyset \neq C\subseteq D$ konvex, dann hei\ss t die Funktion $f\colon D\to\R$ konvex auf $C$, wenn
	\begin{equation*}
		f((1-t)x+ty)\leq(1-t)f(x)+tf(y)
	\end{equation*}
	f\"ur alle $x,y\in C$ und alle $t\in\left[0,1 \right]$ gilt.
\end{Definition}

\begin{Definition}[Epigraph]
\label{def:epi}
 Der \emph{Epigraph} einer Funktion $f : \R^n \to \R$ ist definiert als
 \begin{equation*}
 	\lbrace (x, r)^T \, \vert \, x \in D, r \geq f(x) \rbrace \subseteq \R^{n + 1} =: \text{epi}\, f\qedhere
 \end{equation*}
\end{Definition}
\begin{Bemerkung}[Epigraphcharakterisierung von Funktionen]
	Die obige Definition erlaubt uns nun eine konvexe Funktion als eine Funktion aufzufassen, deren Epigraph eine konvexe Menge ist.
	In der Tat lassen sich die meisten Klassen von Funktionen, die in der Optimierung eine Rolle spielen, durch "aquivalente Charakterisierungen des Epigraphen in Mengensprache fassen.
\end{Bemerkung}
\begin{Definition}[strikte und gleichm"a"sige Konvexit"at]
\label{def:str-convty}
 Gilt f"ur eine Funktion $f \colon \R^n \to \R$
 \begin{equation*}
 	\forall t \in (0, 1) \forall x,y \in D : f((1 - t)x + ty) < (1 - t)f(x) + tf(y)
 \end{equation*}
 so sprechen wir von einer \emph{stikt}, \emph{streng} oder \emph{stark} konvexen Funktion.
 Gilt au"serdem auch
 \begin{equation*}
 	\forall t \in (0, 1) \forall x,y \in D : f((1 - t)x + ty) + \frac{\lambda}{2}\Vert x \Vert^2_2 \leq (1 - t)f(x) + tf(y)
 \end{equation*}
 so sprechen wir auch von einer \emph{gleichm"a"sig} konvexen Funktion zum Parameter $\lambda$. 
\end{Definition}
\begin{Bemerkung}
	Wenn die beiden Konzepte verbal formulieren wollen, so bedeutet \emph{strikt} konvex, dass eine Funktion superlinear gekr"ummt ist.
	\emph{Gleichm"a"sige} Konvexit"at bedeutet dann, dass die Kr"ummung von $f$ mindestens quadratisch ist.
	In der Tat gilt
	\begin{equation*}
		\text{glm. Konvexit"at} \Rightarrow \text{strikte Konvexit"at} \Rightarrow \text{Konvexit"at}
	\end{equation*}
	w"ahrend keine der Umkehrungen gilt.
\end{Bemerkung}

\paragraph{Beispiele konvexer Funktionen}

\begin{enumerate}[label=\emph{\alph*})]
	\item $f(x) = x$ ist konvex auf $\R$, ebenso auch $-f(x)$ und $f$ damit ebenfalls konkav.
	\item $f(x) = x^2$ ist sogar strikt konvex auf $\R$ %nach foliendef sogar glm conv
	\item $f(x) = x^3$ ist nicht konvex auf $\R$ aber gleichm"a"sig konvex auf $C = [0, \infty)$
	\item $f(x) = \exp(\alpha x)$ ist f"ur alle $\alpha$ konvex auf $\R$
	\item $f(x) = -\log(x)$ ist strikt konvex auf $(0, \infty)$
	\item Eine beliebige Norm $\Vert\cdot\Vert$ auf $\R^n$ ist konvex, aber nicht strikt konvex ("Ubungsblatt 1)
	\item Die $\max$-Funktion $f(x) = \max \lbrace x_1, \ldots, x_n \rbrace$ ist konvex auf $\R^n$
\end{enumerate}
%begin characterisations of convex functions
\subsection{Charakterisierungen konvexer Funktionen}
Wir betrachten nun ausreichend sch"one Funktionen, d.\,h., Funktionen f"ur die die Voraussetzungen der folgenden Theoreme gegeben sind.
Dazu z"ahlen insbesondere die differenzierbaren und die stetig differenzierbaren Funktionen.
\begin{Satz}[Charakterisierung erster Ordnung]
\label{chr1O1}
	Seien $D\subseteq\R^n$ offen, $\emptyset\neq\F\subseteq D$ konvex und $f:D\to\R$ differenzierbar auf $D$.
	Dann ist $f$ auf $\F$ konvex genau dann, wenn $\forall x, y \in \F$ gilt:
	\begin{equation}
		f(y) \geq f(x) + \nabla f(x)\tr(y - x)
	\end{equation}
\end{Satz}
%
\begin{Satz}[Charakterisierung erster Ordnung]
\label{chr1O2}
	Seien $D\subseteq\R^n$ offen, $\emptyset\neq\F\subseteq D$ konvex und $f:D\to\R$ differenzierbar auf $D$.
	Dann ist $f$ auf $\F$ strikt konvex genau dann, wenn $\forall x, y \in \F$ gilt:
	\begin{equation}
		f(y) > f(x) + \nabla f(x)\tr(y - x)
	\end{equation}
	\noindent
	Gilt au"serdem f"ur $\mu \in (0, \infty)$
	\begin{equation}
		f(y) - f(x) > \nabla f(x)\tr(y - x) + \mu\Vert y - x \Vert^2
	\end{equation}
	\noindent
	so hei"st $f$ gleichm"a"sig konvex.
\end{Satz}
%
\begin{Satz}[Charakterisierung zweiter Ordnung]
\label{thm:chr2O1}
	Seien $D\subseteq\R^n$ offen, $\emptyset\neq\F\subseteq D$ konvex und $f:D\to\R$ zweimal stetig differenzierbar auf $D$.
	Wenn $\forall x, y \in \F$ gilt:
	\begin{equation}
		\nabla^2 f(x) \succcurlyeq 0
	\end{equation}
	\noindent
	so ist $f$ konvex auf $\F$ und falls $\F$ offen ist, so gilt auch die Umkehrung.
\end{Satz}
%
\begin{Satz}[Charakterisierung zweiter Ordnung]
\label{thm:chr2O2}
	Seien $D\subseteq\R^n$ offen, $\emptyset\neq\F\subseteq D$ konvex und $f:D\to\R$ zweimal stetig differenzierbar auf $D$.
	Wenn $\forall x\in \F$ gilt:
	\begin{equation}
		\nabla^2 f(x) \succ 0
	\end{equation}
	\noindent
	also $\forall x \in \F$ und $0\neq d\in\R^n$
	\begin{equation}
		d\tr\nabla^2 f(x)d > 0
	\end{equation}

	\noindent
	so ist $f$ stikt konvex auf $\F$.
\end{Satz}
%
\begin{Bemerkung}
	Die Umkehrung von \ref{thm:chr2O2} gilt im Allgemeinen nicht.
\end{Bemerkung}
%
\begin{Satz}[Charakterisierung zweiter Ordnung]
\label{thm:chr2O3}
	Seien $D\subseteq\R^n$ offen, $\emptyset\neq\F\subseteq D$ konvex und $f:D\to\R$ zweimal stetig differenzierbar auf $D$.
	Ist $\nabla^2 f(x)$ gleichm"a"sig positiv definit, also wenn $\forall x \in \F$, $d\in\R^n$ und $\beta\in(0,\infty)$
	\begin{equation}
		d\tr\nabla^2 f(x)d > \beta \Vert d\Vert^2
	\end{equation}

	\noindent
	gilt, so ist $f$ gleichm"a"sig konvex auf $\F$. Ist $\F$ offen, so gilt auch die Umkehrung.
\end{Satz}
%
\paragraph{Rechenregeln f"ur konvexer Funktionen}


\begin{enumerate}[label=\emph{\alph*})]
	\item Seien $f_i, i\in[n]$ konvex und $\alpha_i\in\R$.
		  Dann ist auch $\sum_{i=1}^n\alpha_if_i$ konvex.
	\item Seien $f$ konvex, $A\in\R^{m\times n}$ und $b\in\R^m$.
		  Dann ist auch $g(x):=f(Ax+b)$ konvex.
	\item Seien $J$ eine beliebige auch unendliche Indexmenge und $f_j$ konvex f"ur alle $j\in J$.
		  Dann ist auch $\max_{j\in J}f_j(x)$ konvex.
	\item Seien $g(x,y)$ konvex in $x$ und $y$ und die Menge $C$ konvex.
		  Dann ist auch $f(x) := \min_{y\in C}g(x,y)$ konvex.
\end{enumerate}
\section{Globale Lösungen und Eindeutigkeit}

\begin{Satz}[Lokale L"osungen sind global]
F\"ur eine konvexe Optimierungsaufgabe ist jede lokale L\"osung eine globale L\"osung, und die L\"osungsmenge von \eqref{eq:P}
\begin{equation*}
	\mathcal{S}=\{ x\in\F ~\vert~ f(x)\leq f(y)~\forall y\in\F\}
\end{equation*}
ist konvex.
\end{Satz}

\begin{proof}
Sei $x^*$ eine lokale L"osung von \eqref{eq:P}, d.\,h. $\exists r > 0$:
\begin{equation}
 \label{eq:lok1}
 \tag{*}
f(x)>f(x^*) \text{ f"ur alle } x \in B(x^*,r) \cap \F \,.
\end{equation}
Sei $y \in \F$ beliebig, $y \neq x^*$. Wir müssen zeigen, dass $f(y)\geq f(x^*)$ ist.

Da $\F$ konvex ist, gilt 
\begin{equation}
x^*+t(y-x^*)= (1-t) x^* + ty \in \F \text{ f"ur alle }  t \in [0,1] \,.
\end{equation}
Nach der Definition von $B(x^*,r)$ ist
\begin{equation*}
x^*+t(y-x^*)= (1-t) x^* + ty \in B(x^*,r) \text{ f"ur alle }  t \in [0,\frac{r}{\norm[y-x^*]}] \,.
\end{equation*}
Wegen \eqref{eq:lok1} gilt
\begin{equation*}
f(x^*)\leq f(x^*+t(y-x^*))=f((1-t)x^*+ty)\overset{\text{f konvex}}{\leq} (1-t)f(x^*)+tf(y) \\
\Rightarrow f(x^*)\leq f(y) \,.
\end{equation*}
Da $y\in \F$, $y \neq x^*$, beliebig war, ist $x^*$ globale Lösung von \eqref{eq:P}.

Seien $x,z \in \mathcal{S}$, $t \in [0,1]$ beliebig.
Es gilt $f(x)=f(z)$.
\begin{equation*}
f((1-t)x+tz)\overset{\text{f konvex}}{\leq} (1-t)f(x)+tf(z)\overset{f(x)=f(z)}{=} f(x)
\end{equation*}
Also ist auch $(1-t)x+tz \in \mathcal{S}$, d.\,h. $\mathcal{S}$ ist konvex.
\end{proof}

\begin{Satz}[Eindeutigkeit von L"osungen]
Die zul"assige Menge $\F$ des Problems \eqref{eq:P} sei nichtleer und konvex, und die Zielfunktion $f$ sei strikt konvex auf $\F$. Hat \eqref{eq:P} eine L"osung $x^*$ , dann ist $x^*$ eindeutig bestimmt und strikte globale L"osung von \eqref{eq:P}.
\end{Satz}

\begin{proof}
Ist $y \in \F$ ebenfalls L"osung von \eqref{eq:P}, dann sind $x^*$ und $y$ globale L"osungen, also $f(x^*)=f(y)$.
Aus $\F$ konvex folgt $z:=\frac{1}{2}(x^*+y) \in \F$. \\
Wäre $y\neq x^*$:
\begin{equation*}
f(z)=f(\frac{1}{2}(x^*+y)) \overset{\text{f strikt konvex}}{<} \frac{1}{2}f(x^*) + \frac{1}{2}f(y)\overset{f(x^*)=f(y)}{=} f(x^*)
\end{equation*}
Das steht aber im Widerspruch zur Optimalit"at von $x^*$ dar. Also ist $y=x^*$, und für alle $y\neq x^*$ gilt $f(y)<f(x^*)$.
\end{proof}

\begin{Bemerkung}
	Über die \textbf{Existenz einer L\"osung} haben wir hier noch keine Aussage getroffen! Es kann also sein, dass es keine Lösung gibt.
\end{Bemerkung}

\section{Existenz von L"osungen}

Nach dem \textbf{Satz von Weierstraß} nimmt eine stetige Funktion $f : D \rightarrow \R, D \subseteq \R^n$ auf einer kompakten Menge $K \subseteq D$ ihr Supremum und ihr Infimum an. Beim Problem
	\begin{gather*} 
  		\tag{P}
  			\begin{aligned}
    			\min_x
    			& & & f(x) \\
    			\text{s.t.}
    			& & & x\in \F
  			\end{aligned}
	\end{gather*}
ist die zulässige Menge $\F$ im Allgemeinen aber nicht kompakt.\\
Betrachtet man Niveaumengen, erhält man aus dem Satz von Weierstraß jedoch sofort ein Existenzkriterium für eine Lösung von \eqref{eq:P}.
\begin{Definition}(Niveaumenge)
Ist $f : D \rightarrow \R, D \subseteq \R^n$, eine Funktion und $\alpha \in \R$, dann heißen die Mengen
\begin{align*}
N(f,\alpha)={x \in D | f8x)\leq \alpha}
\end{align*}
\textbf{Niveaumengen} zum Niveau $\alpha$ der Funktion $f$.
\end{Definition}

\begin{Satz}
Ist beim Problem \eqref{eq:P} die Zielfunktion $f$ stetig auf $\F$, und ist für ein $w \in \F$ die Menge 
\begin{align*}
N(f, f (w))\cap \F = {x \in \F | f(x) \leq f(w)}
\end{align*}
kompakt, dann gibt es (mindestens) eine globale Lösung von \eqref{eq:P}.
\end{Satz}

\begin{proof}
Sei $N := N (f, f(w)) \cap \F$. Nach dem Satz von Weierstraß gibt es ein $x^* \in N$ mit $f(x^*) \leq f(x)$ für alle $x \in N$ . Für $x \in \F \setminus N$ ist $f(x) > f(w) \geq f(x^*)$. Damit gilt
\begin{align*}
f (x^*) \leq f(x) \forall x \in \F \,,
\end{align*}
d.\,h., $x^*$ ist globale Lösung von \eqref{eq:P}.
\end{proof}

\begin{Lemma}
Die zulässige Menge des Problems \eqref{eq:P} sei nichtleer und abgeschlossen, die Zielfunktion $f$ sei stetig auf $\F$, und es gelte
\begin{align*}
lim_{\norm[x] \rightarrow \infty, x \in \F} f(x) = + \infty .
\end{align*}
Dann ist für beliebiges $w \in \F$ die Menge $N(f, f(w))\cap \F$ kompakt,d. h., es gibt (mindestens) einen globalen Minimalpunkt von $f$ auf $\F$.
\end{Lemma}

\paragraph{Beispiel}
	\begin{gather*} 
  			\begin{aligned}
    			\min_{x \in \R^n}
    			& & & f(x)=\frac{1}{2}x^tQx+q^tx+c \\
    			& & & \F=\R^n \text{konvex}
  			\end{aligned}
	\end{gather*}
$Q$ positiv semidefinit $\overset{\text{Nr. 8}}\Rightarrow$ f konvex $\Rightarrow$ Konvexes Optimierungsproblem, jede Lösung ist global und die Lösungsmenge konvex

$Q$ positiv definit $\overset{\text{Nr. 8}}\Rightarrow$ f strikt konvex $\Rightarrow$ Konvexes Optimierungsproblem\\
Lösung ist strikt global und eindeutig bestimmt, wenn sie existiert\\
Existenz einer Lösung: $\frac{1}{2}\underbrace{x^tQx}_{\geq\alpha\norm{x}^2}+\underbrace{q^tx}_{\geq-\norm{q}\norm{x}}+c\geq \alpha\norm{x}^2-\norm{q}\norm{x}-|c|\overset{\norm{x}\rightarrow\infty}{\rightarrow}+\infty$\\
$\Rightarrow$ für beliebige $w\in\R^n$ ist $\N(f,f(w))$ kompakt\\
$\Rightarrow$ es existiert mindestens eine globale Lösung\\
$\Rightarrow$ Das Problem hat eine eindeutig bestimmte globale Lösung\\
%\begin{Satz}
%Sei $D \subset \R^n$ offen, $\F \subset D$ nichtleer und konvex, und $f: D \rightarrow \R$ sei differenzierbar auf $D$. Dann ist $f$ konvex auf $\F$ genau dann, wenn
%\begin{align*}
%f(y) \geq f(x) + \nabla f(x)^T(y - x) \forall x, y \in \F
%\end{align*}
%gilt.
%\end{Satz}
\section*{Optimalitätskriterien}
\begin{itemize}
\item notwendig: Wenn $x^*$ Lösung ist, dann muss das Kriterium erfüllt sein.
\item hinreichend: Ist das Kriterium erfüllt, so muss $x^*$ Lösung sein.
\end{itemize}
In der konvexen Optimierung sind notwendige Bedingungen auch hinreichend.
\begin{Satz}[Optimalitätsbedingung erster Ordnung]
\label{opt1}
Ist $f$ differenzierbar auf $D$, dann ist $x^* \in \F$ globale Lösung der konvexen Optimierungsaufgabe \eqref{eq:P} genau dann, wenn
\begin{align*}
\nabla f(x^*)^T(x - x^*) \geq 0  \forall x \in \F
\end{align*}
oder äquivalent
\begin{align*}
-\nabla f(x^*) \in N(\F, x^*)
\end{align*}
gilt.
\end{Satz}
\begin{proof}
\begin{itemize}
\item[$\Rightarrow$] Sei $x^*$ globales Minimum von $f$ auf $\F$. Sei $x\in \F$ beliebig.
\begin{align*}
\F \text{ konvex } &\Rightarrow x^*+t(x-x^*) \in \F \forall t \in [0,1]\\
&\Rightarrow f(x^*+t(x-x^*)) \geq f(x^*) \forall t \in [0,1] \\
f \text{ ist diff'bar } &\Rightarrow \nabla f(x^*)^T(x-x^*)=\lim_{t\searrow 0}\frac{f(x^*+t(x-x^*))-f(x^*)}{t} \geq 0\,.
\end{align*}
\item[$\Leftarrow$] Es gelte $\nabla f(x^*)^T(x - x^*) \geq 0$.
\begin{align*}
\text{Es ist } f(x) &\geq f(x^*)+\underbrace{\nabla f(x^*)^T(x-x^*)}_{\geq 0} \forall x \in \F \\
\Rightarrow f(x) &\geq f(x^*)  \forall x\in \F \\
\N(\F, x^*)&=\lbrace s\in\R^n | s^T(y-x^*)\leq 0 \forall y\in\F \rbrace \\
-\nabla f(x^*)&\in\N(\F,x^*)\Leftrightarrow -\nabla f(x^*)^T(y-x^*)\leq 0 \forall y\in\F
\end{align*}
Der negative Gradient ist im Minimum $x^*$ eine Normalenrichtung von $\F$ in $x^*$.
\end{itemize}
\end{proof}

Zur Erinnerung: $D \in \R^n$ sei eine offene Menge, und die Funktion $f: D \rightarrow \R$ sei konvex. Für das unrestringierte Problem
	\begin{gather*} 
	\label{eq:PU}
  		\tag{PU}
  			\begin{aligned}
    			\min_x
    			& & & f(x)
  			\end{aligned}
	\end{gather*}
ist $x^* \in D$ genau dann globale Lösung von \eqref{eq:PU}, wenn die Bedingung
\begin{align*}
\nabla f(x^*) = 0_n
\end{align*}
erfüllt ist.

\begin{Beispiel} 
  	\begin{align}
   		\min_{x\in\R^n}
   		& & & f(x)=\frac{1}{2}x^tQx+q^tx+c, Q\text{ symmetisch, positiv semidefinit} \\
   		\nabla f(x)=Qx+q
  	\end{align}
$x^*$ ist Lösung $\Leftrightarrow \nabla f(x^*)=Qx^*+q \Leftrightarrow Qx^*=-q$ \\
Annahme $Q$ positiv definit $\Rightarrow$ $Q$ ist invertierbar $\Rightarrow$ $x^*=-Q^{-1}q$ ist Lösung des Problems
\end{Beispiel} 

Mit einer Matrix $A \in \R^{m\times n}$ und einem Vektor $b \in \R^m$ betrachten wir das Problem
	\begin{gather*}
	\label{eq:PLG}
  		\tag{PLG}
  			\begin{aligned}
    			\min_x
    			& & & f(x) \\
    			\text{s.t.}
    			& & & Ax=b \,.
  			\end{aligned}
	\end{gather*}
Dann ist $x^* \in \F = {x | Ax = b}$ genau dann globale Lösung von \eqref{eq:PLG},
wenn mit einem \textbf{Lagrange-Multiplikator} $\lambda^* \in \R^m$
\begin{align*}
\nabla f(x^*)+A^T \lambda^* = 0_n
\end{align*}
erfüllt ist. Hat $A$ vollen Rang, dann ist $\lambda^*$ eindeutig bestimmt.

\begin{Lemma}
Die Lagrang-Multiplikatoren kommen durch Satz \ref{opt1} zustande. Dazu betrachten wir folgende Gleichung $K(\F,x^*)=\text{kern}(A)=\lbrace d|Ad=0\rbrace$, $\F=\lbrace x|Ax=b\rbrace$.
\end{Lemma}
\begin{proof}
\begin{itemize}
\item[$\subseteq$ :]Sei $z\in K(\F,x^*)$, dann gilt $z=\alpha(y-x^*)$, $\alpha\geq 0$, $y\in\F$
$\Rightarrow Az=A[\alpha(y-x^*)]=\alpha A(y-x^*)=\alpha[Ay-x^*]=\alpha[b-b]=0$
\item[$\supseteq$ :] Sei $Az=0$. Mit $y:=z+x^*$ gilt $Ay=Az+Ax^*=b \Rightarrow y\in\F$.
$z=y-x^*\in K(\F,x^*)$
\end{itemize}
\end{proof}
Der Kern ist Unterraum, dann gilt: $N(\F,x^*)=(kern A)^\perp=im A^T=\lbrace A^t\lambda|\lambda\in\R^m$.\\
Die Optimalitätsbedingung ist dann:
\begin{align*}
-\nabla f(x^*)\in N(\F,x^*)=im A^T=\lbrace A^t\lambda|\lambda\in\R^m \\
\Leftrightarrow 0=A^T \lambda^*+\nabla f(x^*) \text{ mit } \lambda^*\in\R^m
\end{align*}

\begin{Beispiel} 
  	\begin{align}
   		\min_{x\in\R^n}
   		& & & f(x)=\frac{1}{2}x^TQx+q^Tx+c, Q\text{ symmetisch, positiv semidefinit} \\
   		s.t. &&& Ax=b, A\in\R^{m\times n}
  	\end{align}
$x^*$ ist Lösung $\Leftrightarrow \exists \lambda^*\in\R^m: \nabla f(x^*A^T\lambda^*)=0, Qx^*+q+A^T\lambda^*=0$.
Lösungssystem:
\begin{align*}
Qx^*+A^T\lambda^*&=-q \\
A^x*&=0
\end{align*}

\end{Beispiel} 

Definieren wir die \textbf{Lagrange-Funktion} $L: D \times \R^m \rightarrow \R$ durch
\begin{align*}
L(x, \lambda) = f(x) + \lambda^T (Ax - b) \,,
\end{align*}
dann sind
\begin{align*}
\nabla_x L(x, \lambda) = \lambda f(x) + A^tT \lambda \text{und} \nabla_\lambda L(x, \lambda) = Ax - b\,.
\end{align*}
Also ist $x^* \in D$ genau dann Lösung des Problems \eqref{eq:PLG}, wenn ein $\lambda^* \in \R^m$ existiert, sodass
\begin{align*}
\nabla L(x^* , \lambda ^*) = \begin{matrix}
\nabla_x L(x^*, \lambda ^* )\\
\nabla_\lambda L(x^*, \lambda^* )
\end{matrix}
= 0_{n+m}
\end{align*}
gilt.

Mit Matrizen $A \in \R^{m\times n}$, $G \in \R^{p\times n}$ und Vektoren $b \in \R^m$, $r \in \R^p$ betrachten wir das Problem
	\begin{gather*}
	\label{eq:PL}
  		\tag{PL}
  			\begin{aligned}
    			\min_x
    			& & & f(x) \\
    			\text{s.t.}
    			& & & Ax=b \\
    			& & & Gx \leq r    			\,.
  			\end{aligned}
	\end{gather*}
Eine Restriktion heißt in einem Punkt $x^* \in \F = {x | Ax = b, Gx  \leq r}$ \textbf{aktiv}, wenn sie mit Gleichheit erfüllt ist. Wir bezeichnen mit
\begin{align*}
J(x) = {1  \leq j  \leq p | \langle g^j , x \rangle = r_j}
\end{align*} 	
die Indexmenge der in $x$ aktiven \textbf{Ungleichungsrestriktionen}.

\subsection{Karush-Kuhn-Tucker-Beding (KKT-Bed.)}

Ein Punkt $x^* \in \F$ ist genau dann Lösung des Problems \eqref{eq:PL}, wenn es
\textbf{Lagrange-Multiplikatoren} $\lambda ^* \in \R m$ und $\mu^* \in \R^p$ gibt, für welche die
\textbf{Multiplikatorenregel}
\begin{align*}
\nabla f(x^* ) + A^T \lambda^* + G^T \mu^* = 0 \,,
\end{align*}
die \textbf{Vorzeichenbedingung}
\begin{align*}
\mu^* \geq 0_p
\end{align*}
und die \textbf{Komplementaritätsbedingung}
\begin{align*}
(\mu^*)^T (Gx^* - r) = 0_p
\end{align*}
erfüllt sind. Sind die Vektoren $a^i , i = 1, . . . , m, g^j , j \in J(x^*)$, linear unabhängig, dann sind die Multiplikatoren eindeutig bestimmt.

Wir definieren für das Problem \eqref{eq:PL} die Lagrange-Funktion $L : D \times \R^m \times \R^p \rightarrow \R$ durch
\begin{align*}
L(x, \lambda, \mu) = f(x) + \lambda^T(Ax - b) + \mu^T(Gx - r)\,.
\end{align*}
Das System bestehend aus den Optimalitätsbedingungen und den Nebenbedingungen heißt \textbf{Karush-Kuhn-Tucker-System}:
\begin{align*}
\nabla_x L(x^* , \lambda ^* , \mu ^* ) = \nabla f(x^*) + A^T\lambda ^* + G^T \mu ^* &= 0_n \\
\nabla_\lambda L(x^*, \lambda ^*, \mu ^*) = Ax^* - b = 0_m \\
\nabla_\mu L(x^* , \lambda ^* , \mu ^* ) = Gx^* - r  \leq 0_p \\
\mu^* \geq 0_p \\
(\mu^*)T (Gx^* - r) = 0_p\,.
\end{align*}



\chapter{Optimierungsverfahren: Grundlagen}
\begin{itemize}
\item Nichtlineare Optimierungsprobleme können i.\,d.\,R. nicht analytisch gelöst werden $\Rightarrow$ \textbf{numerische Lösung}
\item Berechne für das Problem \eqref{eq:P} ausgehend von einem Startpunkt $x^{(0)}$ eine Folge ${x^{(k)}}, k = 1, 2,\ldots, $ mit dem Ziel, dass die Folge gegen die Lösung $x^*$ konvergiert
\item Naheliegendes Ziel bei den Iterationen:
\begin{align*}
f(x^{(k+1)}) < f (x^{(k)}) \,,
\end{align*}
solche Verfahren nennt man \textbf{Abstiegsverfahren}, da die Folge der Funktionswerte $\lbrace f(x^{(k)})\rbrace_{k \in N}$ eine absteigende Folge ist
\item Wenn möglich startet man mit einem zulässigen Punkt $x^{(0)} \in \F$ und versucht, dass $x^{(k)} \in \F$ für alle $k \in N$ gilt - \textbf{Verfahren zulässigen Punkte}
\item Abstiegsverfahren benutzen zur Berechnung von $x^{(k+1)}$ eine \textbf{Abstiegsrichtung} $d^{(k)}$ mit der Eigenschaft
\begin{align*}
f(x^{(k)} + \sigma_k d^{(k)}) < f(x^{(k)})
\end{align*}
für eine hinreichend kleine \textbf{Schrittweite} $\sigma_k$
\item Die Schrittweite $\sigma_k$ ist so zu bestimmen, dass man eine möglichst große Abnahme des Zielfunktionswertes erhält
\item Neuer Iterationspunkt: $x^{(k+1)} = x^{(k)} + \sigma_k d^{(k)}$
\end{itemize}

Mit einer \textbf{differenzierbaren Zielfunktion} $f: \R^n \rightarrow \R$ betrachten wir das unrestringierte Optimierungsproblem
	\begin{gather*} 
	\label{eq:PU}
  		\tag{PU}
  			\begin{aligned}
    			\min_x
    			& & & f(x)
  			\end{aligned}
	\end{gather*}
Zur Herleitung und Untersuchung numerischer Verfahren zur Lösung des Problems \eqref{eq:PU} benötigen wir die folgende Standard-Voraussetzung:
(V) Mit gegebenem $x^{(0)} \in \R^n$ ist die Niveaumenge
\begin{align*}
N_0 = N (f, f(x^{(0)})) = {x \in \R^n | f (x)  \leq f(x^{(0)})}
\end{align*}
kompakt.
Damit existiert eine globale Lösung $x^*$ von \eqref{eq:PU} und es gilt 
\begin{align*}
\nabla f(x^* ) = 0_n \,.
\end{align*}

\begin{Lemma}
Die Funktion $f : \R^n \rightarrow \R$ sei differenzierbar in $x$. Weiter sei $d \in \R^n$ mit
\begin{align*}
\nabla f(x)^Td < 0\,.
\end{align*}
Dann gibt es ein $\sigma > 0$ mit $f(x + \sigma d) < f (x)$ für alle $\sigma \in ]0, \sigma[$.
\end{Lemma}
\begin{proof}
$f$ ist differenzierbar in $x$:
\begin{align*}
\nabla f(x)^Td=\lim_{\sigma\searrow 0 \frac{f(x+\sigma d)-f(x)}{\sigma}}
\end{align*}
Nach Voraussetzung ist $\nabla f(x)^T<0$.
\begin{align*}
&\Rightarrow f(x+\sigma d)-f(x)< 0\\
&\Leftrightarrow f(x+\sigma d)<f(x) \text{für hinreichend kleines } \sigma \,.
\end{align*}
\end{proof}
Ein solcher Vektor $d$ heißt \textbf{Abstiegsrichtung} von $f$ im Punkt $x$.

\paragraph{Beispiele}
\begin{enumerate}[label=\emph{\alph*})]
\item Ist $f : \R^n \rightarrow \R$ differenzierbar in $x$ und ist $\nabla f(x) \neq 0_n$ , dann gilt
\begin{align*}
\nabla f(x)^T (-\nabla f(x)) = - \norm[\nabla f(x)]^2< 0 \,,
\end{align*}
d.\,h., der \textbf{negative Gradient} ist eine Abstiegsrichtung in $x$.
\item Ist $A$ eine beliebige positiv definite $n \times n$-Matrix, dann ist die Richtung
\begin{align*}
d = -A-1 \nabla f(x)
\end{align*}
eine Abstiegsrichtung in $x$.
\end{enumerate}

\section{Allgemeines Abstiegsverfahren mit Schrittweitensteuerung}
\begin{itemize}
\item[1.] Wähle einen Startpunkt $x^{(0)} \in \R^n$ und setze $k = 0$.
\item[2.] Ist $\nabla f(x^{(k)}) = 0_n$, dann stoppe das Verfahren.
\item[3.] Berechne eine Abstiegsrichtung $d^{(k)}$, eine Schrittweite $\sigma_k > 0$ mit
\begin{align*}
f (x^{(k)} + \sigma_k d^{(k)}) < f (x^{(k)})
\end{align*}
und setze $x^{(k+1)} = x^{(k)} + \sigma_k d^{(k)}$.
\item[4.] Setze $k = k + 1$ und gehe zu 2.
\end{itemize}

Ist Voraussetzung (V) erfüllt, dann ist $\N_0$ beschränkt, und $f$ ist auf $\N_0$ beschränkt. Da für das allgemeine Abstiegsverfahren
\begin{align*}
f (x^{(k+1)}) < f(x^{(k)})
\end{align*}
ist, folgt $x^{(k)} \in \N_0$ für alle $k \in \mathbb N$. Daher sind die Folgen ${x^{(k)}}$ und
${f(x^{(k)}}$ beschränkt.
\subsection*{Abbruchkriterien}
Das Abbruchkriterium $\nabla f(x^{(k)}) = 0_n$ in Schritt 2 ist nur für theoretische Zwecke sinnvoll. Praktisch gibt man \textbf{Toleranzen} $\varepsilon_i > 0$ vor, wobei man $\varepsilon_i = 10^{-p}$ wählt, wenn das Ergebnis auf $p$ Stellen genau sein soll. \textbf{Typische Abbruchkriterien} sind:
\begin{itemize}
\item $f(x^{(k)}) - f (x^{(k+1)})  \leq \varepsilon_1 \max\lbrace 1, |f (x^{(k)})|\rbrace$
\item $ \norm{x^{(k+1)} - x^{(k)}}  \leq \varepsilon_2 \max\lbrace 1, \norm{x^{(k)}}\rbrace$ mit $\varepsilon_2 = \sqrt{\varepsilon_1}$
\item $\norm{\nabla f(x^{(k)})}  \leq \varepsilon_3 \max\lbrace1, |f (x^{(k)})|\rbrace$ mit $\varepsilon_3 =\sqrt{\varepsilon_1}$
Da die Abbruchkriterien nicht immer erfüllt werden können, sollte man zusätzlich eine \textbf{maximale Iterationszahl} vorgeben.
\end{itemize}

Wir betrachten zunächst das Lösen \textbf{unrestringierter, quadratischer Optimierungsprobleme}
der Form
\begin{gather*} 
	\label{eq:QU}
  		\tag{QU}
  			\begin{aligned}
    			\min_x
    			& & & f(x)=\frac{1}{2}x^TQx+q^Tx \,.
  			\end{aligned}
	\end{gather*}
Hierfür verwenden wir das \textbf{Gradientenverfahren}, welches auf eine Arbeit von \textbf{Louis Augustin Cauchy} aus dem Jahr \textbf{1847} zurückgeht.
Für Probleme vom Typ \eqref{eq:QU} geben wir eine Abstiegsrichtung $d^{(k)}$ und eine \textbf{Schrittweitenstrategie} $\sigma_k$ an, welche wir in das allgemeine Abstiegsverfahren von Folie 6-6 einsetzen. Wir beweisen anschließend die \textbf{Konvergenz} des resultierenden Verfahrens.

Der \textbf{negative Gradient $\nabla f(x)$} ist nicht nur eine Abstiegsrichtung der Zielfunktion im Punkt $x$, sondern definiert sogar die \textbf{Richtung des steilsten Abstiegs}
wie folgendes Resultat zeigt.
\begin{Lemma}
Sei $f: \R^n \rightarrow \R$ in $x$ differenzierbar mit $\nabla f(x) \neq 0_n$. Dann ist
\begin{align*}
\overline{d}=-\frac{\nabla f(x)}{\norm{\nabla f(x)}}
\end{align*}
Lösung des Optimierungsproblems
\begin{gather*} 
  			\begin{aligned}
    			\min_{d \in \R^n}
    			& & & \nabla f(x)^Td \\
    			s.t. 
    			& & & \norm{d}=1
  			\end{aligned}
	\end{gather*}
\end{Lemma}

\begin{proof}
Für beliebiges $d\in\R^n$ ist $\nabla f(x)^T d \geq -\norm{\nabla f(x)} \norm{d}$ (Chauchy-Schwarz). Daher ist $\nabla f(x)^T d \geq \underbrace{-\norm{\nabla f(x)}}_{\text{Untere Schranke der ZF}} \forall d\in\R^n, \norm{d}=1$.\\
Für $\overline{d}$ gilt:
\begin{align*}
\nabla f(x)^Td=-\frac{f(x)^T \nabla f(x)}{\norm{\nabla f(x)}}=-\frac{\norm{\nabla f(x)}^2}{\norm{\nabla f(x)}}=-\norm{\nabla f(x)}\,.
\end{align*}
Damit ist die Behauptung bewiesen.
\end{proof}
\chapter{Das Gradientenverfahren für quadratische Optimierungsprobleme}

\section{Problemstellung und Abstiegsrichtung}

Mit einer \textbf{differenzierbaren Zielfunktion}\ $f:\mathbb{R}^n \rightarrow \mathbb{R}$ betrachten wir das \textbf{unrestringierte Optimierungsproblem}
\begin{gather}
\label{eq:P}
\tag{PU}
\begin{aligned}
\min_{x\in \mathbb{R}^n}
& & & f(x)
\end{aligned}
\end{gather}
\begin{Lemma}
	Die Funktion $f:\mathbb{R}^n \rightarrow \mathbb{R}$ sei differenzierbar in $x$. Weiter sei $d \in \mathbb{R}^n$ mit $\nabla f(x)^Td < 0$. Dann gibt es ein $\bar\sigma > 0$ mit $f(x + \sigma d) < f(x)$ f\"ur alle $\sigma \in\ ]0, \bar\sigma [$. Ein solcher Vektor $d$ heißt \textbf{Abstiegsrichtung} von $f$ im Punkt $x$.
\end{Lemma}

\section{Allgemeines Abstiegsverfahren mit Schrittweitensteuerung}

\begin{enumerate}
	\item Wähle einen Startpunkt $x(0) \in \mathbb{R}^n$ und setze $k = 0$.
	\item Ist $\nabla f(x(k)) = 0_n$, dann stoppe das Verfahren.
	\item Berechne eine \textbf{Abstiegsrichtung} $d(k)$, eine \textbf{Schrittweite} $\sigma k > 0$ mit\\
	$f(x(k)+\sigma k d(k)) < f(x(k))$ und setze $x(k+1) = x(k) + \sigma kd(k)$ .
	\item Setze $k = k + 1$ und gehe zu $2$.
\end{enumerate}

\section{Das Gradientenverfahren}

Suchrichtung: $d^{(k)} = -\nabla f(x^{(k)})$
\begin{itemize}
	\item Wähle einen Startpunkt $x^{(0)} \in R^n$ und setze k = 0.
	\item Ist $\nabla f(x^{(k)}) = 0_n$ , dann stoppe das Verfahren.
	\item Berechne eine effiziente Schrittweite $\sigma_k$ (bspw. Armijo) und setze
	$x^{(k+1)} = x{(k)} - \sigma_k \nabla f(x^{(k)})$ .
	\item Setze $k = k + 1$ und gehe zu 2.
\end{itemize}
Der negative Gradient ist die eindeutig bestimmte Lösung des quadratischen Problems
\begin{equation}
	\min_{d\in\R^n} \underbrace{f(x^{(k)}) + \nabla f(x^{(k)})^Td}_{Taylor-Approximation 1. Ordnung} + \underbrace{1/2 d^Td}_{\underbrace{1/(2\sigma)||x^{(k+1)} - x^{(k)}||^2}_{Abstand  x^{(k+1)} zu  x^{(k)}}}
\end{equation}
$x^{(k+1)} = x^{(k)} + \sigma d$\\
$d = (x^{(k+1)}-x^{(k)})/\sigma$
\\
\\	
für $Q=I_n$\\
$q = \nabla f(x^{(k)})$\\
$Qd + q = 0$\\
$Id + \nabla f(x^{(k)}) = 0$   <=> $d=-\nabla f(x^{(k)})$

\subsection{Funktionenklassen}	

$\mathcal{F}_L^{k,l} (R^n) \Rightarrow$
Menge aller konvexen Funktionen	$\mathcal{F}:R^n \Rightarrow R$, die k mal stetig differenzierbar sind und deren l-te ABleitung lipschitz-stetig mit Konstante L ist, d.h.

\begin{equation}
||f^{(l)} (x) - f^{(l)} (y)|| \leq L ||x-y|| \forall x,y \in R^n
\end{equation}

\begin{itemize}
	\item Offensichtlich ist $k \geq l$
	\item $\mathcal{F}_L^{k_1,l} \leq \mathcal{F}_L^{k_2,l}, k_1 \geq k_2$
	\item $f_1 \in \mathcal{F}_{L_1}^{k,l}, f_2 \in \mathcal{F}_{L_2}^{k,l}, \alpha, \beta \geq 0$
\end{itemize}

$\Rightarrow \alpha f_1 + \beta f_2 \in \mathcal{F}_{\alpha L_1 + \beta L_2}^{k,l}$\\


$\mathcal{F}^k \Rightarrow$ Menge aller konvexen Funktionen, die k mal stetig differenzierbar sind.

Eigenschaft von $\mathcal{F}^1$: Für $f \in \mathcal{F}^1$ gilt:\\
\begin{equation}
	f(y) \geq f(x+ \nabla f(x)^T (y-x)
\end{equation}\\

Eigenschaften von $\mathcal{F}^{1,1}_L$: Für $f \in \mathcal{F}^{1,1}_L$ gilt:\\
\begin{itemize}
	\item $f(y) \leq f(x) + \nabla f(x)^T (y-x) + L/2 ||x-y||^2$
	\item $f(x) + \nabla f(x)^T (y-x) + 1/(2L) || \nabla f(x) - \nabla f(y)||^2 \leq f(y)$
	\item $ 1/L || \nabla f(x) - \nabla f(y)||^2 \leq (\nabla f(x) - \nabla f(y))^T (x-y)$
	\item $(\nabla f(x) - \nabla f(y))^T (x-y) \leq L ||x-y||^2$
	\item $ \alpha f(x) + (1-\alpha) f(y) \geq f(\alpha x + (1-\alpha)y) + (\alpha(1-\alpha))/(2L) || \nabla f(x) - \nabla f(y)||^2$
	\item $\alpha f(x) + (1-\alpha) f(y) \geq f(\alpha x + (1-\alpha)y) + \alpha (1-\alpha) L/2 ||x-y||^2$
\end{itemize}

\begin{Definition}
Eine stetig differenzierbare Funktion f heißt gleichmäßig konvex mit Konvexitätsparameter $\mu > 0 (f \in S_\mu^1)$, wenn $f(y) \geq \nabla f(x)^T (y-x) + (1/2) \mu ||y-x||^2 $
\end{Definition}\mbox{}\\ Wir definieren den Raum $S_{\mu,L}^{k,l}$ analog zu$ \mathcal{F}_L^{k,l} $\\
Eigenschaften:
$f_1 \in S_{\mu_1}^1, f_2 \in S_{\mu_2}^1, \alpha, \beta \geq 0 $

$\Rightarrow  \alpha f_1 + \beta f_2 \in S^{1}_{\alpha mu_1 + \beta mu_2}$


\begin{Lemma}
	Eine zweimal stetig differenzierbare Funktion f gehört zu $S^{2}_{\mu}\\
	 \Leftrightarrow f^n(x) \geq \mu*I_n  \forall x \in R^n$\\
Für $f \in S_{\mu,L}^{2,1}$ gilt:\\
$\mu I_n \leq f''(x) \leq L I_n$
\end{Lemma}

\begin{Definition}
$Q = L/\mu$ ist die Kondition der Funktion f.
\end{Definition}\mbox{}\\Bei kleinen Q führen kleine Änderungen im Problem zu kleinen Änderungen im Funktionswert.

	
\subsection{Konvergenz des Gradientenverfahrens}

Nachfolgend wird die Konvergenz des Gradientenverfahrens für (gleichmäßig) konvexe, stetig differenzierbare und in der Ableitung Lipschitz-stetige Funktionen gezeigt.\\\\
Die Lipschitz-Konstante $L$ folgt der Ungleichung zwischen der Norm der Divergenz-Differenzen zweier Punkte und der Norm zwischen diesen zwei Punkten (Wiederholung):

\begin{equation*}
  \norm{\nabla f(x) - \nabla f(y)} \leq L\,\norm{x-y} \text{ , } \quad \forall x,y \in \mathcal{N}_0
\end{equation*}
Sei die Schrittweitenstrategie in den folgenden Beweisen $\sigma_k = \sigma$ konstant gewählt.
\begin{Satz}[Konvergenz des Gradientverfahrens für konvexe Funktionen]
\label{thm:konvergenz_grad_verfahren_konvex}
	Sei $f: \mathbb{R}^n \rightarrow \mathbb{R}$ konvex und stetig differenzierbar und die Ableitung $\nabla f$ Lipschitz-stetig mit Lipschitzkonstante $L>0$. Dann gilt für das Gradientenverfahren mit konstanter Schrittweite $0<\sigma<\frac{1}{L}$
  \begin{equation*}
    f(x^{(k)})-f(x^\ast) \leq \frac{\norm{x^{(0)}-x^\ast}^2}{2\sigma k} \text{.}
  \end{equation*}
\end{Satz}
\begin{proof}
  Aus der Lipschitz-Stetigkeit der Ableitung $\nabla f$ folgt
  \begin{equation*}
      f(y) \leq f(x) + \nabla f(x)^\top (y-x) + \frac{L}{2} \norm{y-x}^2_2 \text{ , } \forall x,y \text{.}
  \end{equation*}
  Seien $x$ und $y$ die Punkte der $k$-ten bzw. $(k-1)$-ten Iteration des Gradientenverfahrens, d.h.
  \begin{align*}
     & x = x^{(k)} \text{ ,}\\
    \text{sowie} \qquad & y = x^{(k+1)} = x^{(k)} - \sigma \nabla f(x^{(k)}) \text{.}\\
  \end{align*}
  Aus der Ungleichung ergibt für diese zwei gewählten Punkte:\\
    \begin{gather*}
      			\begin{aligned}
              f(x^{(k+1)}) & \leq f(x^{(k)}) + \nabla f(x^{(k)})^\top \left(x^{(k+1)}-x^{(k)}\right) + \frac{L}{2} \norm{x^{(k+1)}-x^{(k)}}^2_2 \\
              & =f(x^{(k)}) + \nabla f(x^{(k)})^\top \left(-\sigma \nabla f(x^{(k)})\right) + \frac{L}{2} \norm{-\sigma \nabla f(x^{(k)})}^2_2 \\
              & = f(x^{(k)}) + -\sigma \norm{\nabla f(x^{(k)})}^2_2 + \frac{\sigma ^2 L}{2} \norm{\nabla f(x^{(k)})}^2_2 \\
              & = f(x^{(k)}) - \sigma \left(1 - \frac{\sigma L}{2}\right)\norm{\nabla f(x^{(k)})}^2_2 \\
              & \leq f(x^{(k)}) - \frac{\sigma}{2} \norm{\nabla f(x^{(k)})}^2_2 \qquad \qquad \text{  ,   da } 0 < \frac{\sigma L}{2} \leq \frac{1}{2} \\
      			\end{aligned}
    	\end{gather*}
      Es gilt demnach
      \begin{equation*}
        f(x^{(k+1)}) < f(x^{(k)}) \text{ , } \qquad \forall k \quad (\text{solange} \nabla f(x^{(k)}) \neq 0)\text{.}
      \end{equation*}
      Aufgrund der Konvexität, sowie der stetigen Differenzierbarkeit von $f$ gilt
      \begin{equation*}
        f(y) \geq f(x) + \nabla f(x)^\top (y-x) \text{ , }\qquad \forall x,y \text{.}
      \end{equation*}
      Sei nun $x=x^{(k)}, y=x^\ast$, so folgt
      \begin{equation*}
        f(x^{(k)}) \leq f(x^\ast) - \nabla f(x^{(k)})^\top \left(x^\ast - x^{(k)}\right) \text{.}
      \end{equation*}
      Eingesetzt in die Ungleichung aus der Lipschitz-Stetigkeit ergibt sich
      \begin{gather*}
        			\begin{aligned}
                f(x^{(k+1)}) &  \leq f(x^{(k)}) - \frac{\sigma}{2} \norm{\nabla f(x^{(k)})}^2_2 \\
                & \leq f(x^\ast) + \nabla f(x^{(k)})^\top \left(x^{(k)}-x^\ast\right) - \frac{\sigma}{2} \norm{\nabla f(x^{(k)})}^2_2\\
                & = f(x^\ast) + \frac{1}{2\sigma} \bigg[ \underbrace{\norm{x^{(k)}-x^\ast}^2_2-\norm{x^{(k)}-x^\ast}^2_2}_{=0} - \sigma^2 \norm{\nabla f(x^{(k)})}^2_2 \\
                & \qquad + 2\sigma \nabla f(x^{(k)})^\top \left(x^{(k)}-x^\ast\right) \bigg] \\
                & = f(x^\ast) + \frac{1}{2\sigma} \bigg[ \norm{x^{(k)}-x^\ast}^2_2- \bigg\{ \norm{x^{(k)}-x^\ast}^2_2 + \sigma^2 \norm{\nabla f(x^{(k)})}^2_2 \\
                & \qquad - 2\sigma \nabla f(x^{(k)})^\top \left(x^{(k)}-x^\ast\right) \bigg\} \bigg] \\
                & = f(x^\ast) + \frac{1}{2\sigma} \left[ \norm{x^{(k)}-x^\ast}^2_2 - \norm{x^{(k)}-\sigma \nabla f(x^{(k)})-x^\ast}^2_2\right] \\
                & = f(x^\ast) + \frac{1}{2\sigma} \left[ \norm{x^{(k)}-x^\ast}^2_2 - \norm{x^{(k+1)}-x^\ast}^2_2\right] \text{.}\\
        			\end{aligned}
      	\end{gather*}
        Diese Ungleichung lässt sich schreiben als
        \begin{equation*}
          f(x^{(k+1)})-f(x^\ast) \leq \frac{1}{2\sigma} \left[ \norm{x^{(k)}-x^\ast}^2_2 - \norm{x^{(k+1)}-x^\ast}^2_2\right] \text{.} \end{equation*}
        Die Summe dieses Ausdruckes über die Iterationsschritte 1 bis $k$ ist eine Teleskopsumme
        \begin{gather*}
          			\begin{aligned}
                  \sum_{i=1}^k f(x^{(i)})-f(x^\ast) & \leq \sum_{i=1}^k \frac{1}{2\sigma} \left[ \norm{x^{(i-1)}-x^\ast}^2_2 - \norm{x^{(i)}-x^\ast}^2_2\right] \\
                  & = \frac{1}{2\sigma} \left[ \norm{x^{(0)}-x^\ast}^2_2 - \underbrace{\norm{x^{(k)}-x^\ast}^2_2}_{\geq 0}\right] \\
                  & \leq \frac{1}{2\sigma} \norm{x^{(0)}-x^\ast}^2_2 \text{.}\\
          			\end{aligned}
        	\end{gather*}
        Weiterhin gilt Aufgrund der Monotonie von $f(x^{(i)})$
        \begin{equation*}
          \sum_{i=1}^k f(x^{(i)})-f(x^\ast) \geq \sum_{i=1}^k f(x^{(k)})-f(x^\ast) = k \left[ f(x^{(k)})-f(x^\ast) \right] \text{.}
        \end{equation*}
        Demnach folgt aus der Ungleichung der Teleskopsumme die zu beweisende Ungleichung
        \begin{equation*}
          f(x^{(k)})-f(x^\ast) \leq \frac{1}{2\sigma k} \norm{x^{(0)}-x^\ast}^2_2
        \end{equation*}
\end{proof}
\noindent
Sofern eine Vorgabe vorliegt, wie groß die Differenz zwischen den beiden Funktionswerten des $k$-ten Interationsschrittes und der Minimalstelle maximal sein darf, um das Verfahren abzubrechen, kann eine Abschätzung der Anzahl an Iterationsschritten $k$ durchgeführt werden. \\
Sei $\epsilon > 0$, so gilt
\begin{gather*}
        \begin{aligned}
          & f(x^{(k)})-f(x^\ast)  \leq \frac{1}{2\sigma k} \norm{x^{(0)}-x^\ast}^2_2  = \epsilon\\
          \Leftrightarrow \qquad & k  = \frac{1}{2\sigma \epsilon} \norm{x^{(0)}-x^\ast}^2_2 \\
          \Rightarrow \qquad & k = \mathcal{O}(\frac{1}{\epsilon})\\
        \end{aligned}
  \end{gather*}

\begin{Satz}[Konvergenz des Gradientverfahrens für gleichmäßig konvexe Funktionen]
\label{thm:konvergenz_grad_verfahren_glchm_konvex}
  Sei $f: \mathbb{R}^n \rightarrow \mathbb{R}$ gleichmäßig konvex (mit Parameter $\mu$), stetig differenzierbar und die Ableitung $\nabla f$ Lipschitz-stetig mit Lipschitzkonstante $L>0$. Dann gilt für das Gradientenverfahren mit konstanter Schrittweite $0<\sigma\leq\frac{2}{\mu + L}$
  \begin{equation*}
    \norm{x^{(k)}-x^\ast}^2 \leq \left( 1 - \frac{2\sigma \mu L}{\mu + L}\right)^k \norm{x^{(0)}-x^\ast} \text{ ,}
  \end{equation*}
  sowie für $\sigma = \frac{2}{\mu + L}$
  \begin{gather*}
          \begin{aligned}
            \norm{x^{(k)}-x^\ast} & \leq \left(\frac{Q_f-1}{Q_f+1}\right)^k \norm{x^{(0)}-x^\ast} \\
            f(x^{(k)}) - f(x^ \ast)& \leq \frac{L}{2}\left(\frac{Q_f-1}{Q_f+1}\right)^{2k} \norm{x^{(0)}-x^\ast}^2 \\
          \end{aligned}
    \end{gather*}
    wobei $Q_f = \frac{L}{\mu}$ (Kondition).
\end{Satz}
\begin{proof}
  Sei die Variable $r^{(k)}$ als Norm zwischen dem Punkt des $k$-ten Iterationsschrittes und dem Minimalpunkt eingeführt
  \begin{equation*}
    r^{(k)} = \norm{x^{(k)}-x^\ast} \text{.}
  \end{equation*}
  So folgt für das Quadrat von $r^{(k+1)}$
  \begin{gather*}
          \begin{aligned}
            \left(r^{(k+1)}\right)^2 & = \norm{x^{(k+1)}-x^\ast}^2_2\\
            & = \norm{x^{(k)}-\sigma \nabla f(x^{(k)})-x^\ast}^2_2 \\
            & = \left(r^{(k)}\right)^2 - 2\sigma \underbrace{\nabla f(x^{(k)})^\top}_{= \left[\nabla f(x^{(k)}-\underbrace{\nabla f(x^\ast)}_{=0}\right]^\top} \left[x^{(k)}-x^\ast\right] + \sigma^2 \norm{\nabla f(x^{(k)})}^2_2
          \end{aligned}
    \end{gather*}
    Der Term $\left[\nabla f(x^{(k)})-\nabla f(x^\ast)\right]^\top \left[x^{(k)}-x^\ast\right]$ kann mithilfe der Ausdrücke $\left(r^{(k)}\right)^2$ und $\norm{\nabla f(x^{(k)})}^2_2$ nach unten abgeschätzt werden. Dazu werden folgende Relation benutzt:
    \begin{gather*}
            \begin{aligned}
              \left[\nabla f(x^{(k)})-\nabla f(x^\ast)\right]^\top & \geq \mu \left(r^{(k)}\right)^2 \\
              \left[\nabla f(x^{(k)})-\nabla f(x^\ast)\right]^\top & \geq \frac{1}{L} \norm{\nabla f(x^{(k)})-\nabla f(x^\ast)}^2_2 \\
            \end{aligned}
    \end{gather*}
    Weiterhin gilt für die Konditionszahl der Funktion $f$ die Relation $Q_f = \frac{L}{\mu} \geq 1$. \\
    Somit folgt
    \begin{gather*}
            \begin{aligned}
              & \left[\nabla f(x^{(k)})-\nabla f(x^\ast)\right]^\top & \\
              = \quad & \left(\frac{1}{2}+\frac{1}{2}\right) \left[\nabla f(x^{(k)})-\nabla f(x^\ast)\right]^\top & \\
              \geq \quad & \frac{\mu}{2} \left(r^{(k)}\right)^2 + \frac{1}{2L}\norm{\nabla f(x^{(k)})-\nabla f(x^\ast)}^2_2 & \\
              = \quad & \frac{\mu L}{L+L} \left(r^{(k)}\right)^2 + \frac{1}{L+L}\norm{\nabla f(x^{(k)})-\nabla f(x^\ast)}^2_2 & \text{, nun gilt:} L \geq \mu\\
              \geq \quad & \frac{\mu L}{\mu+L} \left(r^{(k)}\right)^2 + \frac{1}{\mu+L}\norm{\nabla f(x^{(k)})-\nabla f(x^\ast)}^2_2 & \text{ .}\\
            \end{aligned}
    \end{gather*}
    Demnach folgt für die Abschätzung für $\left(r^{(k+1)}\right)^2$
    \begin{gather*}
            \begin{aligned}
              \left(r^{(k+1)}\right)^2 & \leq \left(r^{(k)}\right)^2 - \frac{2\sigma \mu L}{\mu+L} \left(r^{(k)}\right)^2 - \frac{2\sigma}{\mu+L}\norm{\nabla f(x^{(k)})-\nabla f(x^\ast)}^2_2 + \sigma^2 \norm{\nabla f(x^{(k)})}^2_2\\
              & = \left( 1 - \frac{2\sigma \mu L}{\mu+L}\right) \left(r^{(k)}\right)^2 + \sigma \left(\sigma - \frac{2}{\mu + L}\right) \norm{\nabla f(x^{(k)})}^2_2 \text{ .} \\
            \end{aligned}
      \end{gather*}
      Aus der Voraussetzung $0<\sigma\leq\frac{2}{\mu + L}$ lässt sich der zweite Term abschätzen. Setze $k\rightarrow k-1$, so folgt die zu beweisende Ungleichung
      \begin{gather*}
              \begin{aligned}
                \left(r^{(k)}\right)^2 & \leq \left( 1 - \frac{2\sigma \mu L}{\mu+L}\right) \left(r^{(k-1)}\right)^2 + \sigma \underbrace{\left(\sigma - \frac{2}{\mu + L}\right)}_{\leq 0} \norm{\nabla f(x^{(k-1)})}^2_2 \\
                & \leq \left( 1 - \frac{2\sigma \mu L}{\mu+L}\right) \left(r^{(k-1)}\right)^2 \\
                & \leq \left( 1 - \frac{2\sigma \mu L}{\mu+L}\right)^2 \left(r^{(k-2)}\right)^2 \\
                & \dots \\
                & \leq \bigg(1 - \frac{2\sigma \mu L}{\mu+L}\bigg)^k \left(r^{(0)}\right)^2
              \end{aligned}
        \end{gather*}
      Wird nun $\sigma = \frac{2}{\mu + L}$ festgelegt, ergibt sich
      \begin{gather*}
              \begin{aligned}
                \left(r^{(k)}\right)^2 & \leq \bigg(1 - \frac{2 \frac{2}{\mu + L} \mu L}{\mu+L}\bigg)^k \left(r^{(0)}\right)^2 \\
                & = \bigg(\frac{(\mu + L)^2 - 4\mu L}{(\mu + L)^2}\bigg)^k \left(r^{(0)}\right)^2 \\
                & = \bigg(\frac{(\mu - L)^2}{(\mu + L)^2}\bigg)^k \left(r^{(0)}\right)^2 \\
                & = \bigg(\frac{Q_f - 1}{Q_f + 1}\bigg)^{2k} \left(r^{(0)}\right)^2 \\
              \end{aligned}
      \end{gather*}
      Daher gilt nach Wurzelziehen der Ungleichung
      \begin{equation*}
        \norm{x^{(k)}-x^\ast}  \leq \left(\frac{Q_f-1}{Q_f+1}\right)^k \norm{x^{(0)}-x^\ast} \\
      \end{equation*}
      Aus der Lipschitz-Bedingung $f(x^{(k)})-f(x^\ast) \leq \frac{L}{2} \left(r^{(k)}\right)^2$ folgt die letzte zu zeigende Ungleichung
      \begin{gather*}
        \begin{aligned}
          \frac{2}{L} \left[ f(x^{(k)})-f(x^\ast) \right] & \leq \left(r^{(k)}\right)^2 \\
          \Leftrightarrow \qquad  f(x^{(k)})-f(x^\ast) & \leq \frac{L}{2} \left(\frac{Q_f-1}{Q_f+1}\right)^{2k} \norm{x^{(0)}-x^\ast}^2
        \end{aligned}
      \end{gather*}
\end{proof}

\section{Die Richtung des steilsten Abstiegs}

Der \textbf{negative Gradient $-\nabla f(x)$} ist nicht nur eine Abstiegsrichtung der Zielfunktion im Punkt x, sondern definiert sogar die \textbf{Richtung des steilsten Abstiegs} wie folgendes Resultat zeigt.

\begin{Lemma}
	Sei $f:\mathbb{R}^n \leftarrow \mathbb{R} \text{in} x$ differenzierbar mit $\nabla f(x) \neq 0_n$ . Dann ist \\
	\begin{gather}
	\begin{aligned}
	\bar d = \frac{-\nabla f(x)}{||\nabla f(x)||}
	\end{aligned}
	\end{gather}
Lösung des Optimierungsproblems
\begin{gather}
	\begin{aligned}
	\min_{d\in \mathbb{R}^n}
	& & & \nabla f(x)^T d \\
	s.t.
	& & & |d| = 1
\end{aligned}
\end{gather}
\end{Lemma}

\section{Das Schrittweitenverfahren von Armijo}

Seien x und eine Abstiegsrichtung d von f in x gegeben. Weiter sei
$c_1 \in ]0, 1[$ eine von x und d unabhängige Konstante. Zur Berechnung
einer effizienten Schrittweite $\sigma$ soll die Abstiegsbedingung
$f(x + \sigma d) \leq f(x) + c_1 \sigma \nabla f(x)^Td$
erfüllt werden. Damit die Schrittweite nicht zu klein wird, fordert man
zusätzlich mit einer von x und d unabhängigen Konstante $c_2 > 0$, dass

\begin{equation}
\sigma \geq -c_2 (\nabla f(x)^Td) / ||d||^2 
\end{equation}
Gegeben seien von x und d unabhängige Konstanten:
$\delta \in ]0, 1[$, $\gamma > 0$ und $0 < \beta_1 \leq \beta_2 < 1$

\begin{enumerate}
	\item  Wähle eine Startschrittweite $\sigma_0$, für die mit $c_2 = \gamma$ gilt:
	\begin{equation}
	\sigma \geq - \gamma (\nabla f(x)^Td)/||d||^2
	\end{equation}
	Setze $j = 0$.
	\item Ist die Abstiegsbedingung
	 $f(x + \sigma d) \leq f(x) + \delta \sigma_j \nabla f(x)^Td$
	 erfüllt, dann setze $\sigma_A = \sigma_j$ und stoppe das Verfahren.
	\item Wähle $\sigma_j+ 1 \in [\beta_1 \sigma_j , \beta_2 \sigma_j ]$.
	\item Setze $j = j + 1$ und gehe zu 2.
\end{enumerate}


Praktisch haben sich folgende Werte bewährt:
\begin{itemize}
\item $\delta$ sollte klein sein; Größenordnung: $\delta = 0.01$
\item$\gamma$ sollte so gewählt werden, dass die Schrittweite 1 und die exakte
Schrittweite nicht ausgeschlossen werden. $\gamma = 10^{-4}$
\item $\sigma_0 = 1$ oder als Approximation der exakten Schrittweite:
\begin{equation}
\sigma_0 = - (\nabla f(x)^T d) / (2 (f(x + d) - f(x) - \nabla f(x)^T d))
\end{equation}
\item Zur Berechnung von $\sigma_j,j \geq 1$ kann man $\beta_1 = \beta_2 = \beta$ wählen:
\begin{equation}
\sigma_j = \beta^j \sigma_0 
\end{equation}
$j = 1, 2, . . .$

\end{itemize}
Oft wählt man $\beta = 1/2$

\subsection{Konvergenz des Verfahrens}

Die theoretische Version des Armijo-Verfahrens konvergiert nach endlich
vielen Schritten, falls die Standard-Voraussetzung erfüllt ist und die
Ableitung der Zielfunktion Lipschitz-stetig ist, d. h., es gibt ein $L > 0$ mit
\begin{equation}
|| \nabla f(x) - \nabla f(y)|| \leq L ||x - y|| \forall x,y \in N_0
\end{equation}
Bei einer praktischen Implementierung mit der diskutierten
Parameterwahl ist jedoch endliche Konvergenz nicht sichergestellt.
Daher sollte eine maximale Iterationszahl vorgegeben werden. Sollte
keine Konvergenz eintreten, kann das Verfahren mit anderen Parametern
neugestartet werden.

\paragraph{Beispiel zur Lipschitz-Stetigkeit}\mbox{}\\
\\
$f(x_1,x_2) = (x_1)^2 + (x_2)^2 $\\
$\nabla f(x_1,x_2) = \begin{bmatrix}2x_1\\2x_2\end{bmatrix}$\\
$\nabla f(x_1,x_2)$ ist $\infty$ mal stetig differenzierbar\\

\begin{equation}
\norm{\nabla f(x) - \nabla f(y)} = \norm{2 \begin{bmatrix}x_1\\x_2\end{bmatrix} -2 \begin{bmatrix}y_1\\y_2\end{bmatrix}}\\
= \norm{2 \begin{bmatrix}x_1 -y_1\\x_2 - y_2\end{bmatrix}}\\
\leq 2 \norm{x-y}
\end{equation}
$\nabla f$ ist lipschitz-stetig mit L=2\\
Die Lipschitz-Stetigkeit kann als ein Verhältnis zwischen Funktionswerten und Argumenten interpretiert werden.

\section{Gradientenverfahren für quadratische Probleme}
Wir betrachten zunächst das Lösen \textbf{unrestringierter, quadratischer Optimierungsprobleme} der Form
\begin{gather}
\label{eq:P}
\tag{QU}
\begin{aligned}
\min_{x\in \mathbb{R}^n}
& & & f(x) = \frac{1}{2}x^TQx+q^Tx
\end{aligned}
\end{gather}
Hierfür verwenden wir das \textbf{Gradientenverfahren}, welches auf eine Arbeit von \textbf{Louis Augustin Cauchy} aus dem Jahr \textbf{1847} zurückgeht.Für Probleme vom Typ (QU) geben wir eine  \textbf{Abstiegsrichtung $d(k)$} und
eine \textbf{Schrittweitenstrategie} $\sigma k$ an, welche wir in das allgemeine Abstiegsverfahren von ? einsetzen. Wir beweisen anschließend die \textbf{Konvergenz} des resultierenden Verfahrens.

\section{Die exakte Schrittweite}
Für einen gegebenen Punkt $x \in \mathbb{R}^n$ und die zugehörige Abstiegsrichtung $d = -\nabla f(x) \neq 0_n$ soll nun eine geeignete \textbf{Schrittweite} berechnet werden. \\
Dazu betrachten wir die Funktion $ \psi:[0, \infty[ \leftarrow \mathbb{R}$ mit $$ \psi(s) = f(x+sd) = \frac{1}{2}s^2d^TQd + s(Qx+q)^Td+\frac{1}{2}x^TQx+q^Tx.$$
Diese Funktion ist in s quadratisch. Es gilt $\psi(0) = f(x)$ und $\phi_0(0) = \nabla f(x)^Td < 0$.\\
D. h., geht man von $x$ aus in Richtung $d$, dann nehmen die Funktionswerte zunächst ab.
Wir wollen die Schrittweite $\sigma_E$ so bestimmen, dass der Zielfunktionswert maximal abnimmt. Dies ist offensichtlich der Fall, wenn $\sigma_E$ der eindeutig
bestimmte Minimalpunkt von $\psi$ ist und damit die Optimalitätsbedingung $\psi(0) = (\sigma_E) = 0$ erfüllt. Dadurch erhalten wir
$$ \sigma_E = \frac{d^Td}{d^TQd}.$$
Man nennt $\sigma_E$ die \textbf{exakte Schrittweite}.

\begin{equation}
\begin{aligned}
	\min_{s\in \mathbb{R}} \psi(s) &= f(x+sd) \\
	f(x+sd) & = \frac{1}{2}(x+sd)^TQ(x+sd)+q^T(x+sd)\\
	& = \frac{1}{2}x^TQd+\frac{1}{2}s^2d^TQd+sx^TQd+q^Tx+sq^Td \\
	\psi'(s) &= sd^TQd+(Qx+q)^Td \\
	\psi'(0)&= (Qx+q)^Td \\ 
	&= \nabla f(x)^Td \\
\end{aligned}
\end{equation}
\begin{equation}
\begin{aligned}
	\sigma_E\text{ ist Min von }\psi \Leftrightarrow \psi'(\sigma_E) = 0\\
	\sigma_Ed^TQd=(Qx=q)^Td = 0 \\
	\Leftrightarrow \sigma_E = - \frac{(Qx+q)^Td}{d^TQd} = \frac{d^Td}{d^TQd}\\
	-\nabla f(x) = d
\end{aligned}
\end{equation}

\section{Gradientenverfahren für quadratische Probleme}
\begin{enumerate}
	\item Wähle einen Startpunkt $x(0) \in \mathbb{R}^n$ und setze $k = 0$.
	\item Ist $\nabla f(x(k)) = Qx^{(k)} + q = 0_n$, dann stoppe das Verfahren.
	\item Berechne zu $d(k) = -\nabla f(x(k))$, die exakte Schrittweite $$\sigma_k = \frac{(d^{k})^Td^{(k)}}{(d^{(k)})^TQd^{(k)}}$$
	und setze $x^{(k+1)} = x^{(k)} + \sigma_kd(k)$.
	\item  Setze $k = k + 1$ und gehe zu $2$.	
\end{enumerate}
\subsection{Konvergenz des Verfahrens}
\begin{Theorem}
Ist die Matrix $Q$ positiv definit, dann konvergiert die mit dem Gradientenverfahren für quadratische Optimierungsprobleme berechnete Folge {$x^{(k)}$} für jeden Startpunkt $x(0) \in \mathbb{R}^n$ gegen den eindeutig
bestimmten Minimalpunkt $x^{*}$ von $f$.
\end{Theorem}
\begin{proof}
$x^{*}$ sei globaler Minalmalpunkt von $f \Leftrightarrow \nabla f(x^{*}=Qx+q= 0_n) $
\begin{align*}
\Rightarrow (i) d^{(k)} &= - \nabla f(x^{(k)})\\
 &= -q-Qx^{(k)} = Q(x^{*}-x^{(k)})\\
Qx^{*} &\leftrightarrow x^{*} - x^{k} = q^{-1}d^{(k)}\\
\intertext{Wir definieren $F(x) = (x-x^{*})^TQ(x-x^{*})$}
F(x^{(k)} - F(x^{(k+1)}) &= 2[x^{(k)}-x^{(k+1)}]^TQ[x^{(k)}-x^{(k)}]-[x^{(k+1)}-x^{(k)}]^T[x^{(k+1)}-x^{(k)}]\\
\intertext{Wegen $x^{(k+1)}-x^{(k)} =\sigma_kd^{(k)}$ und $(i)$:}
F(x^{(k)})-F(x^{(k+1)}) &= -2\sigma_k(d^{(k)})^T\underbrace{QQ^{-1}}_{=I_n}d^{(k)}-\sigma_k(d^{(k)})^TQd^{(k)} \\
&= 2\sigma_k(d^{(k)})^Td^{(k)}-\sigma^2_k(d^{(k)})^TQd^{(k)}\\
\intertext{Mit $\sigma_k = \frac{(d^{(k)})^Td^{(k)}}{(d^{(k)})^T-F(x^{(k+1)})}$ folgt}
F(x^{(k)})-F(x^{(k+1)}) &= 2\cdot\frac{[(d^{(k)})^Td^{(k)}]^2}{(d^{(k)})^TQd^{(k)}}-\frac{[(d^{(k)})^Td^{(k)}]^2}{(d^{(k)})^TQd^{(k)}}\\
&= \frac{[(d^{(k)})^Td^{(k)}]^2}{(d^{(k)})^TQd^{(k)}}\\
(i):F(x^{(k)}) &= (x^{(k)}-x^{*})^TQ(x^{(k)}-x^{*})\\
&= -Q^{-1}d^{(k)} = -Q^{-1}d^{(k)}\\
&= (d^{(k)})^T(Q^{-1})^TQQ^{-1}d^{(k)}\\
&= (d^{(k)})^TQ^{-1}d^{(k)} \\
\intertext{Damit folgt:}
\frac{F(x^{(k)})-F(x^{(k+1)})}{F(x^{(k)})} &= \frac{(d^{(k)})^Td^{(k)}}{(d^{(k)})^TQd^{(k)}} - \frac{(d^{(k)})^Td^{(k)}}{(d^{(k)})^TQ{-1}d^{(k)}} \\
\intertext{Sei $\lambda_1$ der kleinste und $\lambda_n$ der groesste Ew von $Q$:}
\lambda_1||d||^2 \in d^TQ^{-1}d&\geq \lambda_n||d||^2\\
\intertext{Damit sind $\lambda_n^{-1} $ der kleinste und $\lambda_1^{-1}$ der groesste Ew von $Q^{-1}$ und }
\frac{1}{\lambda_n} ||d||^2 \in d^TQ^{-1}d &\geq \frac{1}{\lambda_1}||d||^2\\
\intertext{Damit }
\frac{F(x^{(k)})-F(x^{(k+1)})}{F(x^{(k)})} &= 1-\frac{F(x^{(k+1)})}{F(x^{(k)})}\leq \frac{\lambda_1}{\lambda_n}\\
\Rightarrow \frac{F(x^{(k+1)})}{F(x^{(k)})} &=1-\frac{\lambda_1}{\lambda_n}\\
\intertext{Mit $L:=1-\frac{\lambda_1}{\lambda_n}$ folg $0\leq L\leq 1$ und}
F(x^{(k+1)})&\leq L\cdot F(x^{(k)})\\
\intertext{Wiederholte Anwendung der Formel ergibt:}
F(x^{(k+1)}) \leq L^k\cdot F(x^{(k)}) &\Leftrightarrow \underbrace{(x^{(k)}-x^{*})^TQ(x^{(k)}-x^{*})}_{\geq \lambda_1\cdot||x^{(k)}-x^{*}||^2}\leq L^k\cdot F(x^{(0)})\\
\Rightarrow ||x^{(k)}-x^{*}|| &\leq \frac{1}{\lambda_1}L^k\cdot F(x^{(k)})\\
\intertext{Sei $\Gamma := \sqrt{L}$, dann ist $0 \leq \Gamma \le 1$ und}
||x^{(k)}-x^{*}|| &\leq \underbrace{(\lambda_1^{-1}F(x^{(0)}))^\frac{1}{2}\cdot \Gamma^k}_{\xrightarrow[]{k \rightarrow \infty} 0} \hfill k=0,1,2,3,\cdots
\end{align*}
Dies zeigt $x^{(k)} \xrightarrow[]{k \rightarrow \infty} x^{*}$
\end{proof}
Das Gradientenverfahren erzeugt \textbf{orthogonale Suchrichtungen}:
$$0 = \psi'(\sigma_k) = \nabla f(x(k) + \sigma_k d(k))Td(k) = \nabla f(x^{(k+1)})^Td^{(k)} = -(d^{(k+1)})^T d^{(k)}$$
Das beeinträchtigt die Konvergenz des Verfahrens.
Die Konvergenzgeschwindigkeit ist abhängig von der \textbf{Kondition} der Matrix $Q$,
$k(Q) = \lambda_n /\lambda_1$.
Je kleiner die Kondition der Matrix $Q$ ist, desto besser ist das Problem konditioniert, desto besser ist die Konvergenz.
\begin{Beispiel}
	Wenn $Q=I_n$\\
	$\rightarrow$ Problem konvergiert in einem Schritt zur Loesung
\end{Beispiel}
\subsection{Das Verfahren konjugierter Gradienten}
Äquivalent zum Lösen des Problems (QU) mit positiv definiter Matrix $Q$
ist das Lösen des linearen Gleichungssystems
$$Qx = -q$$.
Das \textbf{Verfahren konjugierter Gradienten} (engl. conjugate gradient method, daher \textbf{CG-Verfahren}) wurde \textbf{1952} von \textbf{Hestenes und Stiefel} zum Lösen solcher Gleichungssysteme entwickelt.
\begin{itemize}
	\item Schrittweiten: exakt (wie im Gradientenverfahren)
	\item Suchrichtungen: \textbf{Orthogonalisierung} der Richtungen $\nabla f(x^{(k)}$ bzgl.der Matrix $Q$
\end{itemize}
\subsection{Q-orthogonale Vektoren}
\begin{Definition}
	Sei $Q$ eine symmetrische, positiv definite $n x n$-Matrix. Die Vektoren $d^{(0)} , \dots , d^{(k)}$, $k < n$, heißen \textbf{zueinander konjugiert (orthogonal) bezüglich $Q$ oder $Q$-konjugiert ($Q$-orthogonal)}, wenn sie vom
	Nullvektor verschieden sind und $$(d^{(i)})^TQd^{(j)} = \langle Qd^{(i)} \text{, }d^{(j)}\rangle = 0 \text{, } 0 \leq i \le j \leq k$$ gilt.

\end{Definition}

\subsection{CG-Verfahren}
\begin{enumerate}
	\item Wähle einen Startpunkt $x^{(0)} \in \mathbb{R}^n$ , berechne $\nabla f(x^{(0)}) = Qx^{(0)} + q$, $d^{(0)} = -\nabla f(x^{(0)})$, und setze $k = 0$.
	\item Ist $\nabla f(x^{(k)}) = 0_n$ , dann stoppe das Verfahren.
	\item Berechne zu $d^{(k)}$ die exakte Schrittweite \\
	$	\sigma_k =\nabla -\frac{f(x^{(k)})^T d^{(k)}}{(d^{(k)})^TQd^{(k)}}$ und setze
	$x^{(k+1)} = x^{(k)} + \sigma_k d^{(k)}$ .\\
	Berechne die neue Suchrichtung $d^{(k+1)}$ durch
$$\nabla f(x^{(k+1)}) = Qx^{(k+1)}+q = \nabla f(x^{(k)})+\sigma_kQd{(k)}$$
$$\beta_k = \frac{||\nabla f(x^{(k+1)})||^2}{\nabla f(x^{(k)})}\text{, } d^{(k+1)} = -\nabla f(x^{(k+1)}) + \beta_kd^{(k)}$$
	\item Setze $k = k + 1$ und gehe zu Schritt 2.
\end{enumerate}


\subsection{Konvergenz des CG-Verfahren}
\begin{Theorem}
Ist die Matrix $Q \in \mathbb{R}^{n x n}$ positiv definit, dann berechnet das CG-Verfahren in $m \geq n$ Iterationsschritten den eindeutig bestimmten Minimalpunkt $x^{*}$ von $f$. Die berechneten Richtungen $d^{(k)}$ sind $Q$-konjugierte Abstiegsrichtungen von $f$ in $x^{*}$.
\end{Theorem}
\begin{proof}
Satz 4.2.14, Nichtlineare Optimierung (Walter Alt). 
\end{proof}
Wesentlich für die Konvergenz des Verfahrens ist, dass die Folge von
Suchrichtungen orthogonal bzgl. $Q$ und damit linear unabhängig ist. Im
Vergleich zum Gradientenverfahren passt sich das CG-Verfahren durch
die Orthogonalisierung besser an die Zielfunktion an.
\section{Optimierungsverfahren für allgemeine Probleme}

Allgemeine, unrestringierte Probleme mit \textbf nichtlinearer, differenzierbarer Zielfunktion vom Typ
$
			\begin{aligned}
				\min_{x\in\R^n}
				& & & f(x) & & &
			\end{aligned} $ lassen sich mit den im vorigen Kapitel behandelten Verfahren nicht lösen. Der negative Gradient als Richtung des steilsten Abstiegs ist immer noch als Abstiegsrichtung geeignet, allerdings kann keine exakte Schrittweite mehr bestimmt werden.\\
Sei $x^{(k)}$ ein Iterationspunkt mit einer Abstiegsrichtung $d^{(k)}$. Bei hinreichend kleiner Schrittweite $\sigma_k$ gilt:
\begin{equation}
	f(x^{(k+1)}) = f(x^{(k)} + \sigma_kd^{(k)}) < f(x^{(k)})
\end{equation}
Die Folge ${f(x^{(k)})}$ ist streng monoton fallend. Das muss jedoch nicht bedeuten, dass das Problem konvergiert.
\paragraph{Beispiel für nicht konvergierende Probleme}\mbox{}\\
\\
$f(x)= x^2, x^{(0)} = 1, d^{(0)} = -1$ \\

\begin{center}
     \begin{tikzpicture}[xscale=0.5,yscale=0.25]
       \draw[->,thick] (0,0) -- (7,0) node[below]{$x$};
       \draw[->,thick] (1,0) -- (0,0) -- (0,25) node[left]{$y$};
			 \draw (1,1) node{$\circ$} node[below right]{\footnotesize{$(x_0)$}};
			 \draw[blue] plot (\x,{(\x)^2});
     \end{tikzpicture}
		\captionof{figure}{Beispiel fuer nicht konvergierendes Problem bei unguenstig gewaehlter Schrittweite}
\end{center}



\begin{align*}
\intertext{Abstiegsrichtung: $\nabla f(x^{(k)})^T d^{(k)} =  -2x^{(k)} < 0 \, \, \, \, \forall x^{(k)} > 0$}
\intertext{Betrachte Schrittweite $\sigma_k = 1/2^{k+2} \, \, \, \, \forall k \geq 0$}
x^{(k-1)} &= x^k + \sigma_k d^{(k)} \\
&= x^{(k)} - \sigma_k \\
&= x^{(k)} - (1/2)^{(k+2)} \\
&= x^{(0)} - \sum\limits_{i=1}^k (1/2)^{(i+2)} \\
&= 1 - \sum\limits_{i=1}^k (1/2)^{(i+2)} \\
&= 1/2 + (1/2)^{(k+2)}  \xrightarrow[]{k \rightarrow \infty} 1/2
\intertext{Die Folge $x^{(k)}$ konvergiert nicht gegen $x^{*} = 0$.}
\end{align*}


\paragraph{Mögliche Schrittweitenstrategien}

\begin{itemize}
	\item Konstante Schrittweite $\sigma_k = \sigma > 0$
	\item Kleiner werdende Schrittweiten, z.B. $\sigma_k = 1/k$
	\item Exakte Schrittweiten, z.B. $\sigma_k = arg \min {\sigma \geq 0} f(x^{(k)} + \sigma d^{(k)})$
	\item  Armijo-Verfahren
\end{itemize}





\chapter{Untere Schranken für das Gradientenverfahren}

\section{Untere Schranken für $\F_L^{\infty, 1}$}

Im letzten Kapitel wurden obere Schranken für die Konvergenzgeschwindigkeiten betrachtet, jetzt soll es um untere Schranken gehen.

\textit{Modell:}
\begin{align*}
\min_{x \in \R^n} f(x) \quad , f \in \F^{\infty,1}_L
\end{align*}

\textit{Orakel:} Orakel 1. Ordnung, d.h. nur $f(x)$ und $f'(x)$ sind bekannt.

\textit{approximative Lösung:} $\bar{x} \in \R^n : f(\bar{x}) - f^* \leq \epsilon$

\noindent
\textit{Annahme}: eine iterative Methode generiert eine Folge von Testpunkten $\{x^{(k)}\}_k$, für die gilt:
\begin{align*}
x^{(k)} \in x^{(0)} + span\{f'(x^{(0)}), f'(x^{(1)}), …, f'(x^{(k-1)})\}, \quad k \geq 1.
\end{align*}

\begin{Bemerkung}
Diese Annahme ist nicht unbedingt notwendig, macht aber den Beweis einfacher.
\end{Bemerkung}

Sei $L>0$. Dann betrachte folgende Familie von Funktionen (für $x = (x_1,x_2,…,x_n)\in\R^n$):
\begin{align*}
f_k(x) &= \frac{L}{4} \left( \frac{1}{2} \left[ x_1^2 + \sum_{i=1}^{k-1}(x_i-x_{i+1})^2 + x_2^2 \right] - x_1\right)\\
       &= \frac{L}{4} \left( \frac{1}{2} x\tr A_k x - e_1\tr x \right)
\text{mit}\,A_k \in \R^{n \times n}, A_k =
\left(
\begin{array}{c}
\begin{array}{rrrr|c}
2  & -1     &        &    & 0\\
-1 & \ddots & \ddots &    & \\
   & \ddots & \ddots & -1 & \\
   &        &   -1   & 2  & \\
\end{array}\\
\hline
\begin{array}{rrrr|c}
\phantom{-}0  & \phantom{-1}  & \phantom{-1} & \phantom{-1} & 0 \\
\end{array}
\end{array}
\right)\raisebox{0.5\normalbaselineskip}{%
$\left.\rule{0pt}{2.5\normalbaselineskip}\right\}k\,\text{Zeilen}$}
\end{align*}

\paragraph*{Eigenschaften dieser Funktion}
\begin{itemize}
\item $f_k''(x) = \frac{L}{4} A_k$
\item $s\tr f_k''(x)s \geq 0 \; \forall s$, denn $f_k''(x)$ ist positiv semidefinit
\item $s\tr f_k''(x)s \leq L \norm{s}_2^2$
\end{itemize}

Daraus folgt $LI_n \geq f_k''(x)$ und somit ist $f_k \in \F^{\infty,1}_L$.

Für das Optimum von $f_k$ muss gelten $f_k'(x^*) = 0 = \frac{L}{4}(A_k x - e_1)$.

Es gibt eine Lösung $x^* \in \R^n$ mit
\begin{align*}
x_i^* = \begin{cases}
1 - \frac{i}{k+1} & \text{für}\, i=1,\dots,k\\
0 & \text{für}\, i=k+1,\dots,n
\end{cases}
\end{align*}

Damit ergibt sich der optimale Funktionswert zu
\begin{align*}
f^* = f_k(x^*)
&= \frac{L}{4} \left( \frac{1}{2} (x^*)\tr A_k x^* - e_1\tr x^* \right)\\
&= \frac{L}{4} \left( \frac{1}{2} (x^*)\tr \underbrace{( A_k x^* - e_1)}_{=0} - \frac{1}{2}e_1\tr x^* \right)\\
&= \frac{L}{4} \left( -\frac{1}{2}e_1\tr x^* \right)\\
&= -\frac{L}{8}x_1^*\\
&= -\frac{L}{8} \left( 1- \frac{1}{k+1} \right).
\end{align*}

Mit der Abschätzung $\displaystyle \sum_{i=1}^{k}i^2 = \frac{k(k+1)(2k+1)}{6} \leq \frac{(k+1)^3}{3}$ folgt nun:
\begin{align*}
\norm{x}^2 &= \sum_{i=1}^{n} (x_i^*)^2\\
&\leq \sum_{i=1}^{k} (x_i^*)^2\\
&= \sum_{i=1}^{k} (1-\frac{i}{k+1})^2\\
&= k - \frac{2}{k+1} \sum_{i=1}^{k}i + \frac{1}{(k+1)^2} \sum_{i=1}^{k}i^2\\
&\leq k - \frac{2}{k+1} \cdot \frac{k(k+1)}{2} + \frac{1}{(k+1)^2} \cdot \frac{(k+1)^3}{3}\\
&= \frac{k+1}{3}.
\end{align*}

\begin{Definition}
Sei $\R^{k,n} = \{x\in\R^n | x_i = 0, k+1 \leq i \leq n \}$ der Teilraum des $\R^n$, bei dem die ersten $k$ Komponenten $\neq 0$ sein können.
\end{Definition}

Es gilt für alle $x \in \R^{k,n}$, dass $f_p(x) = f_k(x)$  für $p = k,…,n$ (da $x_p=0$ für $p<k$).

Sei nun $p$ fest mit $1 \leq p \leq n$.

\begin{Lemma}
Sei $x^{(0)} = 0$. Dann gilt für jede Folge $\{x^{(k)}\}_{k=0}^p$ mit
\begin{align*}
x^{(k)} \in \L_k = x^{(0)} + span\{ f'(x^{(0)}), f'(x^{(1)}), …, f'(x^{(k-1)})\}
\end{align*}
dass $\L_k \subseteq \R^{k,n}$.
\end{Lemma}
\begin{proof} (per Induktion über k)
\begin{itemize}
\item[IA:]
Wegen $x^{(0)}=0$ gilt $f'_p(x^{(0)}) = -\frac{L}{4}e_1 \in \R^{1,n}$. Daraus folgt $\L_1 \subseteq \R^{1,n}$.
\item[IV:]
Sei $\L_k \subseteq \R^{k,n}$ für $k \leq p$.
\item[IB:]
Es gelte $\L_{k+1} \subseteq \R^{k+1,n}$ für $k \leq p$.
\item[IS:]
Da $A_k$ tridiagonal ist, gilt für jedes $x \in \R^{k,n}$ dass $f'_p(x) \in \R^{k+1,n}$. Daraus folgt $\L_{k+1} \subseteq \R^{k+1,n}$.
\end{itemize}
\end{proof}

\begin{Lemma}
Für jede Folge $\{x^{(k)}\}_{k=0}^p$ mit $x^{(0)}=0$ und $x^{(k)} \in \L_k$ gilt:
\begin{align*}
f_p(x^{(k)}) \geq f_k^*.
\end{align*}
\end{Lemma}

\begin{proof}
Aus $x^{(k)} \in \L_k$ folgt $f_p(x^{(k)}) = f_k(x^{(k)}) \geq f_k^*$.
\end{proof}

%%%%%%%%%%

\begin{Theorem}
Für jedes $k$ mit $1\leq k \leq \frac{1}{2} (n - 1)$ und jedes $x^{(0)} \in \R^n$ existiert eine Funktion $f \in \F_L^{\infty, 1}$, so dass für jede iterative Methode 1. Ordnung gilt:
\begin{align*}
f(x^{(k)}) - f^* &\geq \frac{3L \norm{x^*-x^{(0)}}^2}{32 (k + 1)^2}\\
\norm{x^{(k)} - x^*}^2 &\geq \frac{1}{8} \norm{x^* - x^{(0)}}^2,
\end{align*}
wobei $x^*$ das Minimum von $f(x)$ und $f^* = f(x^*)$ ist.
\end{Theorem}

\begin{proof}
Offensichtlich sind Methoden dieses Typs invariant unter Verschiebung.

Wähle o.B.d.A. $x^{(0)} = 0 \in \R^n$.
Sei $k$ fest gewählt. Es wird nun der Algorithmus angewandt auf:
\begin{align*}
f(x) &= f_{2k + 1}(x)\\
x^* &= x_{2k+1}^*\\
f^* &= f_{2k+1}^*
\intertext{Aus vorherigem Lemma folgt:}
f(x^{(k)}) &= f_{2k + 1}(x^{(k)}) = f_k(x^{(k)}) \geq f_k^*.
\intertext{Wegen $x^{(0)} = 0$ folgt:}
\frac{f(x^{(k)}) - f^*}{\norm{x^{(0)} - x^*}^2} &\geq \frac{\frac{L}{8}(-1 + \frac{1}{k+1} + 1 - \frac{1}{2k + 2})}{\frac{1}{3}(2k + 2)}
= \frac{\frac{L}{8}\cdot\frac{1}{2}(\frac{1}{k + 1})}{\frac{2}{3}(k + 1)}
=\frac{3 L}{32 (k + 1)^2}
\intertext{2. Ungleichung:}
\lVert x^{(0)} - x^* \rVert^2 &\ge \sum_{i = k + 1}^{2k + 1}(x_i)^2
= \sum_{i = k + 1}^{2k + 1}(1 - \frac{i}{2k + 2})^2\\
&=k + 1 - \frac{1}{k + 1} \cdot \sum_{i = k + 1}^{2k + 1} i + \frac{1}{4(k + 1)^2} \sum_{i = k + 1}^{2k + 1} i^2.
\intertext{Es gilt:}
\sum_{i = k + 1}^{2k + 1} i^2 &= \frac{1}{6}((2k+1)(2k+2)(4k+3)-k(k+1)(2k+1))\\
&=\frac{1}{6}((2k+1)(k+1)(2(4k+3)-k))\\
&=\frac{1}{6}(2k+1)(k+1)(7k+6).
\intertext{Damit ergibt sich}
\norm{x^{(k)} - x^*}^2 &\geq k+1 - \frac{1}{k+1}\frac{(3k+2)(k+1)}{2} + \frac{1}{24(k+1)^2} \cdot (2k+1)(k+1)(7k+6)\\
&=k+1 - \frac{3k+2}{2} + \frac{(2k+1)(7k+6)}{24(k+1)}\\
&=\frac{2k+1)(7k+6)}{24(k+1)}-\frac{k}{2}\\
&=\frac{(2k+1)(7k+6)-12(k+1)k}{24(k+1)}\\
&=\frac{14k^2 + 12k + 7k + 6 - 12k^2 - 12 k}{24(k+1)}\\
&=\frac{2k^2 + 7k + 6}{24(k+1)}\\
&\geq \frac{2k^2 + 7k + 6}{16(k + 1)^2} \norm{x^{(k)} - x^*}^2\\
&\geq \frac{1}{8} \norm{x^{(0)} - x^*}^2.
\end{align*}
\end{proof}

\begin{Bemerkung}
Die Schritte sind nur gültig, solange $k \leq \frac{1}{2} (n - 1)$ gilt (wobei $n$ die Dimension ist).
Komplexitätsschranken von diesem Typ heißen uniform in der Dimension der Variablen.
Sie geben Informationen darüber, wie schnell ein Gradientenverfahren bei großen Problemen konvergiert.
Aber: ohne endlichdimensionale Argumente gibt es keine Verbesserung.
\end{Bemerkung}

\begin{Bemerkung}
Die Konvergenz zum optimalen Punkt kann beliebig langsam sein.
Dies behebt man mit der Funktionsklasse $\mathcal{S}$ der gleichmäßig konvexen Funktionen.
\end{Bemerkung}

\section{Untere Schranken für $\mathcal{S}_{\mu, L}^{\infty, 1}$}
\textit{Modell:}
\begin{align*}
\min_{x \in \R^n}; f \in \mathcal{S}_{\mu, L}^{\infty, 1} \quad , \mu > 0
\end{align*}

\textit{Orakel:} Orakel 1. Ordnung ($f(x)$ und $f'(x)$ sind gegeben).

\textit{approximative Lösung $\bar{x}$:}
\begin{align*}
f(\bar{x}) - f^* &\leq \varepsilon\\
\norm{\bar{x} - x^*}^2 &\leq \varepsilon
\end{align*}

Wir haben bisher keine Annahme über die Dimension $n$ des Definitionsbereichs von ${f:\R^n \rightarrow \R}$ getroffen.
Da das einfacher geht, betrachten wir jetzt den unendlich-dimensionalen Definitionsbereich $\R^\infty$.

Erinnerung: $\R^\infty \equiv l_2$ ist der Raum aller Folgen $\lbrace x_i \rbrace_{i = 1}^\infty$ mit endlicher Norm, d.h.
\begin{align*}
\norm{x}^2 = \sum_{i = 1}^{\infty} x_i^2 < \infty.
\end{align*}

Sei $\mu > 0$ und $L > \mu$ gegeben, d.h. $Q_f = \frac{L}{\mu} > 1$.
\begin{align*}
f_{\mu, Q_f} &= \frac{\mu(Q_f - 1)}{8}(x\tr Ax-2e_1\tr x)+\frac{\mu}{2}\norm{x}^2
\quad \text{mit}\,A =
\begin{pmatrix}
2 & -1 & \\
-1 & \ddots & \ddots\\
 & \ddots & \ddots\\
\end{pmatrix}\\
f''(x) &= \frac{\mu(Q_f - 1)}{4} \cdot A + \mu \cdot I,
\end{align*}
wobei $I$ der 1-Operator im $\R^\infty$ ist.

Es gilt
\begin{align*}
4 I \succeq A \succeq 0
\end{align*}
und damit folgt
\begin{align*}
\mu  Q_f I = (\mu (Q_f - 1) + \mu) I \succeq f''(x) \succeq \mu I
\Longrightarrow f \in \mathcal{S}_{\mu, \mu Q_f}^{\infty, 1}
\end{align*}
Die Kondition von $f$ ist dann $\frac{\mu \cdot Q_f}{\mu} = Q_f$.

Wir wollen nun das Minimum $x^*$ bestimmen, es muss also gelten:
\begin{align*}
\nabla f_{\mu, Q_f}(x) &= \left( \frac{\mu(Q_f-1)}{4} A + \mu I\right) x - \frac{\mu(Q_f - 1)}{4}e_1 \overset{!}{=} 0\\
&\Leftrightarrow \left(A + \frac{4}{Q_f -1}\right)x \overset{!}{=} e_1.
\end{align*}
In Koordinatenschreibweise:
\begin{align*}
2\frac{Q_f + 1}{Q_f -1}x_1 - x_2 &= 1\\
2_{k+1} - 2 \frac{Q_f + 1}{Q_f - 1}x_k + x_{k-1} &= 0 \quad \text{für}\, k = 2, 3,…
\end{align*} % ??

\begin{Theorem}
Für jedes $x^{(0)} \in \mathbb{R}^\infty$ und jede Konstante $\mu > 0$ und $Q_f > 1$ existiert eine Funktion $f \in \S_{\mu, \mu Q_f}^{\infty, 1}$, sodass für jedes iterative Optimierungsverfahren, welches die obige Annahme erfüllt, gilt:
\begin{align*}
\norm{x^{(k)} - x^*}^2 &\geq \left(\frac{\sqrt{Q_f} - 1}{\sqrt{Q_f} + 1}\right)^{2k} \norm{x^{(0)} - x^*}^2\\
f(x^{(k)}) - f(x^*) &\geq \frac{\mu}{2} \left(\frac{\sqrt{Q_f} - 1}{\sqrt{Q_f} + 1}\right)^{2k} \norm{x^{(0)} - x^*}^2,
\end{align*}
wobei $x^*$ das Minimum von $f$ ist.
\end{Theorem}
\begin{proof}
(entfällt.)
\end{proof}
\begin{Bemerkung}
Für gleichmäßig konvexe Funktionen stimmt diese untere Schranke mit dem Gradientenverfahren überein.
\end{Bemerkung}

\section{Optimale Methoden für das Gradientenverfahren (Nesterov 1983)}
\begin{itemize}
\item Das Standard-Gradientenverfahren ist nicht optimal.
\item Das Standard-Gradientenverfahren will immer die größte Reduktion pro Iteration („greedy“), was häufig nicht die beste Methode ist.
\end{itemize}

\subsection{Nesterov's Accelerated Gradient Method (NAGM)}
\begin{enumerate}
\item Starte mit $x^{(0)} \in \mathbb{R}^n, \alpha_0 \in (0, 1)$, sei $y^{(0)} = x^{(0)}, q = \frac{\mu}{L}$
\item In der k-ten Iteration:
\begin{itemize}
	\item Berechne $x^{(k + 1)} = y^{(k)} - \frac{1}{L} \nabla f(y^{(k)})$
	\item Berechne $\alpha_{k+1} \in (0, 1)$ als Lösung der Gleichung
	\begin{align*}
	\alpha_{(k + 1)}^2 = (1 - \alpha_{k + 1}) \alpha_k^2 + q \alpha_{k + 1}
	\end{align*}
	\item Setze $\beta_k = \frac{\alpha_k(1 - \alpha_k)}{\alpha_k^2 + \alpha_{k + 1}}$
	\item $y^{(k + 1)} = x^{(k + 1)} + \beta_k ( x^{(k + 1)} - x^{(k)})$
\end{itemize}
\end{enumerate}

\begin{Theorem}
Für $\alpha_0 \in (0,1)$ und $\alpha_0^2 L = (1 - \alpha_0)L + \alpha_0 \mu$ generiert NAGM eine Folge $\lbrace x^{(k)} \rbrace$ mit 
\begin{align*}
f(x^{(k)}) - f(x^*) \leq L \cdot \min\Biggl\{\underbrace{\left(1 - \sqrt{\frac{\mu}{L}}\right)^k}_{c^k\rightarrow\log(\frac{1}{\varepsilon})}, \underbrace{\frac{4}{(k + 2)^2}}_{\frac{1}{k^2}\rightarrow\frac{1}{\varepsilon}}\Biggl\} \norm{x^{(0)} - x^*}^2.
\end{align*}
\end{Theorem}
\begin{proof}
(entfällt.)
\end{proof}
\chapter{\textsc{Newton}-Verfahren}
\section{Das \textsc{Newton}-Verfahren}
\textsc{Newton}-Richtung $d^{(k)} = -f''\left(x^{(k)}\right)^{-1}\nabla f\left(x^{(k)}\right)$ \underline{nicht} so berechnen!
Sondern über Gleichungssystem!

\[-A^{-1}\nabla f\left(x^{(k)}\right)\]
\[A = f''\left(x^{(k)}\right)\]

$\rightarrow$ Das zu lösende Gleichungssystem is linear (im Gegensatz zu dem ursprünglichen
nicht-linearen von dem \textsc{Newton}-Verfahren).

$Ax = b \overset{\text{in Matlab}}{\rightsquigarrow}$ \texttt{x = A\textbackslash b} (LU-Faktorisierung)

\[d^{(k)} = -f''\left(x^{(k)}\right)^{-1}\nabla f\left(x^{(k)}\right)\]

$f''(x^{(k)})$ positiv definit
$\Rightarrow f''(x^{(k)})^{-1}$ positiv definit
$\Rightarrow d\tr f''(x^{(k)})^{-1}d > 0 \quad \forall d \in \R^n$

Eigenwerte $\lambda_1, \ldots, \lambda_n > 0$, Eigenwerte $\frac{1}{\lambda_1}, \ldots, \frac{1}{\lambda_n} > 0$

\begin{Beispiel}{Newtonschritt\\}
\[\min f(x) = \frac{1}{2}x\tr \begin{pmatrix}
4 & 0 \\
0 & 2
\end{pmatrix}x + \begin{pmatrix}
-4 \\
-2
\end{pmatrix}\tr x + 3,\quad \text{Startpunkt: }
x^{(0)} = \begin{pmatrix}
4 \\
4
\end{pmatrix}\]



\[\nabla f(x) = \begin{pmatrix}
4 & 0 \\
0 & 2
\end{pmatrix} x + \begin{pmatrix}
-4 \\
-2
\end{pmatrix}\]
\[f''(x) = \begin{pmatrix}
4 & 0 \\
0 & 2
\end{pmatrix} \rightsquigarrow f''(x)^{-1} = \begin{pmatrix}
\frac{1}{4} & 0 \\
0 & \frac{1}{2}
\end{pmatrix}\]

\begin{enumerate}
 \item Ist der Gradient schon 0?
 \item[] $\nabla f\left(x^{(0)}\right) = \begin{pmatrix}
4 & 0 \\
0 & 2
\end{pmatrix}\begin{pmatrix}
4 \\
4
\end{pmatrix} + \begin{pmatrix}
-4 \\
-2
\end{pmatrix} = \frac{12}{6} + 0_2 \rightarrow$ nein!
 \item Berechne $d$ und $x$
 \item[] $d^{(0)} = -f''\left(x^{(0)}\right)^{-1}\nabla f\left(x^{(0)}\right) = -\begin{pmatrix}
\frac{1}{4} & 0 \\
0 & \frac{1}{2}
\end{pmatrix}\begin{pmatrix}
12 \\
6
\end{pmatrix} = -\begin{pmatrix}
3 \\
3
\end{pmatrix}$
\item[] $x^{(1)} = x^{(0)} + d^{(0)} = \begin{pmatrix}
4 \\
4
\end{pmatrix} - \begin{pmatrix}
3 \\
3
\end{pmatrix} = \begin{pmatrix}
1 \\
1
\end{pmatrix}$
  \item Gradient 0?
  \item[] $\nabla f\left(x^{(1)}\right) = \begin{pmatrix}
4 & 0 \\
0 & 2
\end{pmatrix} \begin{pmatrix}
1 \\
1
\end{pmatrix} + \begin{pmatrix}
-4 \\
-2
\end{pmatrix} = \begin{pmatrix}
0 \\
0
\end{pmatrix}$ \checkmark
\item[$\rightarrow$] \textsc{Newton}-Verfahren konvergiert nach einem Schritt für quadratische
Probleme.
\end{enumerate}
\end{Beispiel}

\textsc{Newton}-Verfahren muss gedämpft werden, weil volle Schritte teilweise
in die falsche Richtung gehen könnten (z.\,B. bei $f(x) = \sqrt{1 + x}$):
\begin{center}
\begin{tikzpicture}[y=0.80pt, x=0.80pt, yscale=-1.000000, xscale=1.000000, inner sep=0pt, outer sep=0pt]
\begin{scope}[shift={(-0.53125,-802.625)}]
    \path[draw=black,line join=miter,line cap=butt,line width=0.832pt]
      (0.5406,951.8622) -- (299.4614,951.8622);
    \path[draw=black,line join=miter,line cap=butt,line width=0.588pt]
      (150.5199,952.1022) -- (150.5199,802.6222);
    \path[draw=black,line join=miter,line cap=butt,line width=0.965pt]
      (75.6030,947.4451) -- (75.6030,955.7591);
    \path[draw=black,line join=miter,line cap=butt,line width=0.965pt]
      (225.6030,947.4452) -- (225.6030,955.7592);
    \path[color=black,fill=black,line width=1.095pt] (2.7188,802.8438) .. controls
      (16.0212,872.0668) and (76.9019,924.3750) .. (150.0000,924.3750) .. controls
      (223.0981,924.3750) and (283.9788,872.0668) .. (297.2812,802.8438) --
      (295.8438,802.8438) .. controls (282.5683,871.3055) and (222.3498,923.0000) ..
      (150.0000,923.0000) .. controls (77.6419,923.0000) and (17.3964,871.3169) ..
      (4.1250,802.8438) -- (2.7188,802.8438) -- cycle;
    \path[draw=black,line join=miter,line cap=butt,line width=0.800pt]
      (81.3173,906.9002) -- (247.4874,886.1921) -- (34.8503,869.5246) --
      (285.8732,834.1692) -- (6.5660,814.4713);
    \path[draw=black,line join=miter,line cap=butt,line width=0.794pt]
      (6.7446,813.9662) -- (14.7009,811.4408);
    \path[draw=black,line join=miter,line cap=butt,line width=0.876pt]
      (6.6189,814.3739) -- (13.6578,819.4576);
    \path[draw=black,line join=miter,line cap=butt,line width=0.800pt]
      (282.3376,837.0734) -- (285.3681,834.2955) -- (281.7063,832.2752);
    \path[draw=black,line join=miter,line cap=butt,line width=0.800pt]
      (38.7646,866.4941) -- (35.2291,869.2720) -- (39.1434,872.0499);
    \path[draw=black,line join=miter,line cap=butt,line width=0.800pt]
      (243.8256,889.2225) -- (247.1086,886.1921) -- (243.8256,882.9091);
    \path[fill=black] (64.6498,921.0424) node[above right] (text3136) {$x^{(0)}$};
    \path[fill=black] (69.4141,967.3622) node[above right] (text3140) {$-1$};
    \path[fill=black] (222.6816,967.3622) node[above right] (text3140-7) {$1$};
\end{scope}
\end{tikzpicture}
\end{center}


\section{Das gedämpfte \textsc{Newton}-Verfahren}
\begin{enumerate}
 \item Wähle Startpunkt $x^{(0)} \in \R^n, k := 0$
 \item Ist $\nabla f(x^{(k)}) = 0_n \rightarrow$ Stop
 \item Berechne die \textsc{Newton}-Richtung $d^{(k)} = -f''(x^{(k)})^{-1}\nabla f(x^{(k)})$,
 eine effiziente Schrittweite $\sigma_k$ (z.\,B. \textsc{Armijo}) und setze $x^{(k+1)} = x^{(k)} + \sigma_kd^{(k)}$
 \item Setze $k := 1$ und gehe zu 2.
\end{enumerate}

\begin{Bemerkung}
 $d^{(k)} = -A^{-1}\nabla f(x^{(k)})$\newline
 Gradientenverfahren: $A = I_n$\newline
 \textsc{Newton}-Verfahren: $A = f''(x^{(k)})$
\end{Bemerkung}

\subsection{Konvergenz des gedämpften \textsc{Newton}-Verfahrens}
\begin{itemize}
 \item $f$ sei gleichmäßig konvex und zweimal stetig differenzierbar
 \item $\nabla f$ sei \textsc{Lipschitz}-stetig mit Konstante $L$
 \item Konvexitätsparameter $\mu$
 \item $f''(x)$ sei \textsc{Lipschitz}-stetig mit Konstante $M$, d.\,h.,
 \item[] $\norm{f''(x) - f''(y)}_F \leq M \cdot \norm{x - y}_2 \qquad \forall x, y$
 \item[] $\norm{A}_F := \sum_{i, j} |A_{ij}|^2$ (\textsc{Frobenius}-Norm)
\end{itemize}

\begin{Theorem}
 \[
 f(x^{(k)}) - f(x^*) \leq
 \begin{cases}
 f(x^{(0)}) - f(x^*) - \gamma \cdot k & \text{für } k \leq k_0\\
 \frac{2\mu^3}{M^2} \left(\frac{1}{2}\right)^{2^{k - k_0 + 1}} & \text{für } k > k_0
 \end{cases}
 \]

Seien $\delta, \beta$ die \textsc{Armijo}-Parameter, dann
$\gamma = \frac{\delta \cdot \beta^2 \cdot \eta^2 \cdot \mu}{L^2}$,
$\eta = \min \{1, 3 \cdot (1 - 2 \cdot \delta)\} \cdot \frac{\mu^2}{M}$
und $k_0$ ist die Anzahl der Iterationen bis $\norm{\nabla f(x^{(k_0 + 1)})} < \eta$.
\end{Theorem}

Gegeben sind $\gamma > 0$, $0 < \eta \leq \frac{\mu}{M^2}$. Das gedämpfte \textsc{Newton}-Verfahren konvergiert in zwei Phasen:
\begin{enumerate}
 \item Gedämpfte Phase: $\norm{\nabla f(x^{(k)})} \geq \eta$ und $f(x^{(k + 1)}) - f(x^{(k)}) \leq -\gamma$
 \item[] \textsc{Armijo}-Verfahren wählt hier typischerweise eine Schrittweite, die kleiner als $1$ ist (\glqq ge\-dämpft\grqq).
 \item Reine \textsc{Newton}-Phase: $\norm{\nabla f(x^{(k)})} < \eta$ und
 $\frac{M}{2\mu^2}\norm{\nabla f(x^{(k+1)})} \leq \left(\frac{M}{2\mu^2}\norm{\nabla f(x^{(k)})}\right)^2$
 \item[] \textsc{Armijo}-Verfahren wird sofort mit Schrittweite 1 stoppen (\glqq reine\grqq{} \textsc{Newton}-Phase).
 Sobald diese Phase erreicht ist, wird sie nicht mehr verlassen. \\
 Mit $\eta \leq \frac{\mu}{M^2}$ gilt
 $\frac{2\mu^2}{M} \cdot \left(\frac{M}{2\mu^2}\cdot\eta\right)^2 < \eta$ 
 $\Rightarrow \norm{\nabla f(x^{(k+1)})} < \eta$.
\end{enumerate}

Wir wollen $f(x^{(k)}) - f(x^*) \leq \epsilon$ erreichen. Dazu benötigt man
\[\frac{f(x^{(0)}) - f(x^*)}{\gamma} + \log\bigg(\log\bigg(\frac{\epsilon_0}{\epsilon}\bigg)\bigg)\quad\text{Iterationen.}\]

Um in die reine \textsc{Newton}-Phase zu kommen, werden höchstens $\frac{f(x^{(0)}) - f(x^*)}{\gamma}$ Iterationen benötigt.
In der reinen \textsc{Newton}-Phase ist die Konvergenzrate $\log(\log(\frac{\epsilon_0}{\epsilon}))$, mit $\epsilon_0 = \frac{2\mu^3}{M^2}$.
$\mathcal{O}(\log(\log(\frac{1}{\epsilon})))$ heißt quadratische Konvergenz (nur lokal). Zur Erinnerung: Das Gradientenverfahren hat eine lineare Konvergenz mit $\mathcal{O}(log(\frac{1}{\epsilon}))$.

Implementierungshinweise:
\begin{itemize}
 \item Gleichungssystem lösen (in GNU Octave und Matlab löst \lstinline{A\b} die Gleichung $Ax = b$) anstatt die Matrix zu invertieren.
 \item Startschrittweite $\sigma_0 = 1$ wählen.
 \item $d^{(k)} = -f''(x^{(k)})^{-1} \nabla f(x^{(k)})$ ist eine Abstiegsrichtung, wenn $H := f''(x^{(k)})$ positiv definit ist. In der Implementierung sollte daher
 $H + \gamma \cdot I_n$ mit z.\,B. $\gamma = 10^-5$ verwendet werden, was numerisch stabiler ist, da negative Eigenwerte in der \textsc{Hesse}-Matrix verhindert werden. Gggf.\,kann das Verfahren dadurch eine (oder mehr) Iterationen länger brauchen, da es nicht mehr genau das \textsc{Newton}-Verfahren ist.
\end{itemize}

\FloatBarrier
\subsection{Vergleich des gedämpften \textsc{Newton}-Verfahrens mit dem Gradientenverfahren}
\begin{table}
 \begin{tabularx}{\textwidth}[!Htb]{lXX}
  \toprule
  & \textsc{Newton}-Verfahren & Gradientenverfahren \\
  \midrule
  Speicher & $\mathcal{O}(n^2)$ ($n\times n$ \textsc{Hesse}-Matrix) & $\mathcal{O}(n)$ ($n$-dimensionaler Gradient) \\
  Berechnungen & $\mathcal{O}(n^3)$ (lineares Gleichungssystem lösen, dicht besetzt) & $\mathcal{O}(n)$ (skalieren \& addieren von $n$-dimensionalen Vektoren) \\
  \textsc{Armijo}-Verfahren & $\mathcal{O}(n)$ & $\mathcal{O}(n)$ \\
  Konditionierung & unabhängig von Kondition (zumindest Lokal) & kann stark beeinträchtigen \\
  Stabilität & \glqq etwas anfälliger\grqq & robust \\
  \bottomrule
 \end{tabularx}
 \caption{Vergleich zwischen \textsc{Newton}-Verfahren \& Gradientenverfahren}
\end{table}

Ziel ist es, den Aufwand aus den Berechnungen der \textsc{Hesse}-Matrizen zu reduzieren.

\begin{Satz}
 Sei $A$ eine reelle $n \times n$ Matrix, symmetrisch und positiv definit. Dann wird durch
 \[<x, y>_A = x\tr Ay \qquad \forall x, y \in \R^n\]
 ein Skalarprodukt definiert, und durch
 \[\norm{x}_A = \sqrt{<x,x>_A} = \sqrt{x\tr Ax} \qquad \forall x \in \R^n\]
 eine Norm auf $\R^n$ definiert. Ungleichung von \textsc{Cauchy}-\textsc{Schwarz}: $<x,y>_A \leq \norm{x}_A \cdot \norm{y}_A$.
\end{Satz}

Wir betrachten folgendes Problem:

\begin{equation}
 \begin{aligned}
  \min_{d \in \R^n} & \qquad & \nabla f(x)\tr d\\
  \text{s.\,t.}     & \qquad & \norm{d}_A = 1
  \end{aligned}
  \label{equation_20170608_1}
\end{equation}

\begin{Lemma}
 Sei $f: \R^n \rightarrow \R$ differenzierbar in $x$ mit $\nabla f(x) \neq 0_n$, und $A$ eine symmetrische, positiv definite $n \times n$-Matrix.
 Dann ist
 \[\overline{d} = -\frac{A^{-1} \nabla f(x)}{\norm{A^{-1} \nabla f(x)}_A}\]
 Lösung von \autoref{equation_20170608_1}.
 \label{lemma_20170608_1}
\end{Lemma}

\begin{proof}
 Für $d \in \R^n$ gilt
 
 \begin{equation}
  \begin{aligned}
   \nabla f(x)\tr d &= \nabla f(x)\tr \underbrace{A^{-1}A}_{=I} d = <A^{-1}\nabla f(x), Ad>\\
   &= <A^{-1} \nabla f(x), d>_A\\
   &\overset{\text{\textsc{Cauchy}-\textsc{Schwarz}}}{\geq} -\norm{A^{-1} \nabla f(x)}_A \cdot \norm{d}_A
  \end{aligned}
 \end{equation}

 Daher gilt für $\norm{d}_A = 1$: $\nabla f(x)\tr d \geq -\norm{A^{-1} \nabla f(x)}_A$.
 
 Speziell für $\overline{d}$ gilt: $\nabla f(x)\tr d = \ldots = -\norm{A^{-1} \nabla f(x)}_A$.
\end{proof}

\textsc{Newton}-Verfahren: Setze $A=f''(x)$. Im Gegensatz zum Gradientenverfahren wird in jeder Iteration eine andere Norm
(Metrik) zur Bestimmung des steilsten Abstiegs verwendet. Das gedämpfte \textsc{Newton}-Verfahren zählt daher zur Klasse der
\emph{Variable-Metrik-Verfahren}

\section{Das Quasi-\textsc{Newton}-Verfahren}
Quasi-\textsc{Newton}-Verfahren nutzen Suchrichtungen vom Typ
\[d^{(k)} = -(A^{(k)})^{-1} \nabla f(x^{(k)})\]
Diese Richtungen sind Abstiegsrichtungen wenn $A^{(k)}$ positiv definit ist.
Laut \autoref{lemma_20170608_1} sind dies die Richtungen des steilsten Abstiegs bezüglich der durch $A^{(k)}$ definierten Norm.
Verfahren, die eine solche Suchrichtung verwenden heißen Variable-Metrik-Verfahren.
Wir wollen die Schwierigkeit umgehen, $f''(x)$ zu berechnen und trotzdem noch \glqq schnell\grqq{} konvergieren.
Dazu konstruieren wir ausgehend von einer beliebigen symmetrischen, positiv definiten Matrix $A^{(0)}$ eine Folge $\{A^{(k)}\}$
symmetrischer, positiv definiter Matrizen mit den folgenden Eigenschaften:
\begin{enumerate}
 \item $A^{(k)}$ soll eine Approximation von $f''(x)$ sein.
 \item Der Übergang $A^{(k)} \rightarrow A^{(k+1)}$ soll möglichst einfach sein.
\end{enumerate}

Als Ausgangspunkt dient das Newton-Verfahren hier gilt:
\[A^{(k)}=f''(x^{(k)})\]
\[\Rightarrow\nabla f_{k+1}(x^{(k)})=\nabla f(x^{(k+1)})+A^{(k+1)}\cdot (x^{(k)}-x^{(k+1)})\overset{Taylor}{\approx}\nabla f(x^{(k)})\tag{*}\]
Also wählen wir $A^{(k+1)}$ so, dass:
\[\nabla f_{k+1}(x^{(k)})=\nabla f(x^{(k)})\]
\begin{equation}
\overset{*}{\Rightarrow}A^{(k+1)}\cdot (x^{(k+1)}-x^{(k)}) = \nabla f(x^{(k+1)})-\nabla f(x^{(k)})
\end{equation}
die sogenannte \textbf{Quasi-Newton-Gleichung}.\\

\section{Das BFGS-Verfahren}
In der Praxis hat sich die BFGS-Updateformel für $A^{(k)}$ am besten bewährt,
welche unabhängig von Broyden, Fletcher, Goldfarb und Shanno
entwickelt wurde. \\

\textbf{Herleitung der BFGS-Updateformel:} Sei
\[s^{(k)}=x^{(k+1)}-x^{(k)},\ y^{(k)}=\nabla f(x^{(k+1)}) - \nabla f(x^{(k)})\]
Start mit$A^{(0)}$ symmetrisch, positiv definit (bspw. $A^{(0)}=I_n$).\\ 
Zuerst berechnet man:
\[\tilde{A}^{(k)} = A^{(k)}-\frac{\overbrace{A^{(k)}s^{(k)}\cdot (A^{(k)}s^{(k)})^T}^{dyadisches\ Produkt}}{(s^{(k)})^TA^{(k)}s^{(k)}}\]
Dann gilt:
\[\tilde{A}^{(k)}s^{(k)} = A^{(k)}s^{(k)} - \frac{A^{(k)}s^{(k)}\cdot \cancel{s^{(k)}(A^{(k)}s^{(k)})^T}}{\cancel{(s^{(k)})^TA^{(k)}s^{(k)}}} = A^{(k)}s^{(k)} - A^{(k)}s^{(k)} = 0\]
Falls $A^{(k)}$ symmetrisch und positiv definit ist, ist $\tilde{A}^{(k)}$ symmetrisch und positiv semidefinit. Die Matrix $\tilde{A}^{(k)} - A^{(k)}$ ist symmetrisch und
\[\text{Rang}(A^{(k)}s^{(k)}\cdot (A^{(k)}s^{(k)})^T) = 1\]
da der Rang des dyadischen Produktes stets 1 ist. Daher der Name \emph{symmetrische Rang-1-Modifikation}. Danach erfolgt eine zweite symmetrische Rang-1-Modifikation mit:
\[A^{(k+1)} = \tilde{A}^{(k)}+\gamma_k\cdot w^{(k)}\cdot (w^{(k)})^T\text{ mit }\gamma_k\in\mathbb{R},\ w^{(k)}\in\mathbb{R}^n\]
Ziel: $A^{(k+1)}$ soll positiv definit sein und die Quasi-Newton-Gleichung erfüllen:
\[A^{(k+1)}s^{(k)} = y^{(k)}\]
Es muss wegen $\tilde{A}^{(k)}s^{(k)} =0$ gelten:
\[A^{(k+1)}s^{(k)} = \gamma_k\cdot w^{(k)}\cdot (w^{(k)})^T\cdot s^{(k)} = y^{(k)}\]
Wir wählen dazu:
\[w^{(k)} = y^{(k)},\ \gamma_k=\frac{1}{(y^{(k)})^Ts^{(k)}}\]
Damit $A^{(k+1)}$ positiv definit ist, muss für die Richtung $s^{(k)}$ gelten: ($\rightsquigarrow$bleibt als Voraussetzung)
\begin{equation}
(s^{(k)})^T A^{(k+1)} s^{(k)} = (y^{(k)})^Ts^{(k)} > 0
\end{equation}

\begin{Beispiel}{Voraussetzung bei quadratischen Problemen\\}
Sei $f$ eine quadratische Funktion, $Q$ positiv definit, also $f(x)=\frac{1}{2}x^TQx+q^Tx+c$
\begin{align*}
(y^{(k)})^Ts^{(k)} = (\nabla f(x^{(k+1)}) - \nabla f(x^{(k)}))^T(x^{(k+1)} - x^{(k)})\\
= (Qx^{(k+1)}+q - Qx^{(k)}-q)^T(x^{(k+1)} - x^{(k)})\\
=(x^{(k+1)} - x^{(k)})^TQ(x^{(k+1)} - x^{(k)}) \overset{Q\text{ pos. def.}}{>}0
\end{align*}
\end{Beispiel}

Das heißt obige Bedingung ist bei allen quadratischen Funktion erfüllt. Allgemeiner gilt dies auch für jede gleichmäßig konvex Funktion. (ohne Beweis)\\
Variable-Metrik-Verfahren, bei denen in jedem Iterationsschritt die
Quasi-Newton-Gleichung erfüllt ist, heißen \emph{Quasi-Newton-Verfahren}.\\
Ist $f$ zweimal stetig differenzierbar, so folgt:
\[A^{(k+1)}\cdot (x^{(k)}-x^{(k+1)}) = f''(x^{(k)} +\theta (x^{(k+1)}-x^{(k)}))\cdot (x^{(k+1)}-x^{(k)}) \] mit $0<\theta < 1$, d. h., in Richtung $x^{(k+1)}-x^{(k)}$ verhält sich $A^{(k+1)}$ ungefähr wie $f''(x^{(k+1)}$. Daher enthält $A^{(k+1)}$ Informationen über
die Krümmung von $f$ in $x^{(k+1)}$, die in die Berechnung der folgenden
Suchrichtung einfließen.\\

\textbf{Zusammenfassung:}
\begin{itemize}
\item Durch die Quasi-Newton-Gleichung ist  $A^{(k+1)}$ nicht eindeutig
bestimmt
\item In der Praxis hat sich die BFGS-Updateformel am besten bewährt
\item Beim BFGS-Verfahren berechnet man $A^{(k+1)}$ durch zwei symmetrische
Rang-1-Modifikationen von  $A^{(k)}$
\end{itemize}

\textbf{BFGS-Update-Formel:} (Rang-2-Modifikation)
\begin{equation}
A^{(k+1)} = \underbrace{A^{(k)}- \frac{A^{(k)}s^{(k)}\cdot (A^{(k)}s^{(k)})^T}{(s^{(k)})^TA^{(k)}s^{(k)}}}_{\text{Rang-1-Modifikation}}\underbrace{+ \frac{y^{(k)}(y^{(k)})^T}{(y^{(k)})^Ts^{(k)}}}_{\text{Rang-1-Modifikation}}
\end{equation}

\textbf{Das BFGS-Verfahren:}
\begin{itemize}
 \item Wähle Startpunkt $x^{(0)} \in \R^n, A^{(0)}$ symmetrisch, positiv definit, $k := 0$
 \item Ist $\nabla f(x^{(k)}) = 0_n \rightarrow$ Stop
 \item Berechne die Matrix $A^{(k)}$ nach dem BFGS-Update (für $k \geq 1$), die
Suchrichtung $d^{(k)}$ durch lösen des linearen Gleichungssystems
$A^{(k)} = -\nabla f(x^{(k)})$, die exakte Schrittweite (bei quadratischen Problemen):
\[\sigma_k=-\frac{(\nabla f(x^{(k)}))^Td^{(k)}}{(d^{(k)})^TQd^{(k)}}\]
oder bspw. die Armijo-Schrittweite sonst und setze $x^{(k+1)} = x^{(k)+\sigma_k}d^{(k)}$
\item Setze $k=k+1$.
\end{itemize}

\chapter{Nichtglatte Optimierung}
\section{Einleitung}
Wir betrachten weiter Probleme der Form
\begin{equation}
 \min_{f\in\mathbb{R}^n} f(x)
\end{equation}
wobei $f:\mathbb{R}^n\rightarrow \mathbb{R}$ konvex ist, aber nicht zwingend differenzierbar.

\begin{Beispiel}[Lineare Regression]
Bisher haben wir uns die quadrierte L2-Norm betrachtet:
\begin{equation}
 \min_{x_1,x_2} \sum\limits_{i=1}^m(x_1\xi_i +x_2 +\eta_i)^2.
\end{equation}
Diese wird nun durch die Summe der Fehlerbeträge (L1-Norm) ersetzt:
\begin{equation}
 \min_{x_1,x_2} \sum\limits_{i=1}^m\underbrace{|x_1\xi_i +x_2 +\eta_i|}_{=:f_1(x_1,x_2)}.
\end{equation}
Ein weiteres Beispiel ist es die maximale Abweichung zu minimieren ($\text{L}^\infty$-Norm):
\begin{equation}
 \min_{x_1,x_2} \max_{i=1,\dots,m}\underbrace{|x_1\xi_i +x_2 +\eta_i|}_{=:f_\infty(x_1,x_2)}.
\end{equation}
Sowohl $f_1$ als auch $f_\infty$ sind konvexe Funktionen, die jedoch nicht überall differenzierbar sind.
\end{Beispiel}


\begin{Satz}
 Sei $f:\mathbb{R}^n\rightarrow\mathbb{R}$ eine konvexe Funktionen.
 Dann ist $f$ fast überall (bis auf Lesbesgue-Nullmengen) differenzierbar.
 In der konvexen Optimierung nennt man fast überall differenzierbare Funktionen \textbf{nicht glatt}. 
\end{Satz}

Aufgrund dieses Satzes könnte man vermuten, dass Verfahren für differenzierbare Zielfunktionen im allgemeinen 
konvexen Fall funktionieren. Dies ist jedoch in der Regen nicht richtig.

\section{Probleme}
Die theoretischen Voraussetzungen für die Konvergenz sind nicht erfüllt. Weiter ist das Abbruchkriterium $\nabla 
f(x)=0_n$ nicht anwendbar.

\begin{Beispiel}[Wolfe-Funktion]
\begin{equation}
 f(x_1,x_2) =
 \begin{cases}
  5\cdot\sqrt{9x_1^2+16x_2^2} 	& ,x_1\geq |x_2|\\
  9x_1+16|x_2| 			& ,0<x_1<|x_2|\\
  9x_1+16|x_2|-x_1^9 		& ,x_1\leq 0
 \end{cases}
\end{equation}
Die Funktion ist konvex und stetig, aber auf der Lesbegue-Nullmenge $M=\{(x_1,x_2)|x_1\leq 0,x_2=0\}$ nicht 
differenzierbar. Das eindeutig bestimmte Minimum ist $x^*=(-1,0)\in M$.

Verwendet man das Gradientenverfahren mit exakter Schrittweite, dann konvergiert dieses für jeden Startpunkt 
$x^{(0)}\in S$, wobei $S=\{(x_1,x_2)|x_1>|x_2|>(\frac{9}{16})^2|x_1|\}$ ,gegen den nicht optimalen Punkt 
$\bar{x}=(0,0)$.

Zur Konstruktion effizienter numerischer Verfahren für nicht glatte Probleme muss man möglichst viele 
Eigenschaften konvexer Funktionen ausnutzen.
\end{Beispiel}

\begin{Beispiel}[Transformation von Problemen]
 \begin{align*}
  \min f_\infty(x_1,x_2) \Leftrightarrow & \min z\\
  & \text{s.t.} & -z-\xi_i x_1-x_2 \leq -\eta_i & & i = 1,\dots,m\\
  & & -z-\xi_i x_1+x_2 \leq \eta_i & & i = 1,\dots,m
 \end{align*}
 Diese Transformation erzeugt war ein glattes, lineares Problem, benötigt dafür aber zusätzliche Variablen und 
$2\cdot m$ Nebenbedingungen.
\end{Beispiel}

\section{Operationen mit konvexen Funktionen}
\begin{Lemma}
 Sind $f_1,\dots,f_n$ konvexe Funktionen und $t_1,\dots,t_m$ positive reelle Zahlen und es gibt ein 
$\bar{x}\in\mathbb{R}^n$ mit $f_j(\bar{x})<+\infty,j=1,\dots,m$, dann ist auch 
\begin{equation}
 f:=\sum\limits_{j=1}^m t_j\cdot f_j
\end{equation}
konvex.
\end{Lemma}

\begin{Lemma}
 Ist $J$ eine beliebige Indexmenge, $\{f_j\}_{j \in J}$ eine Familie konvexer Funktionen und es gibt ein $\bar{x} 
\in\mathbb{R}^n$ mit $\sup\limits_{j\in J} f_j(\bar{x}) < +\infty$ dann ist auch $f:=\sup\limits_{j\in J} f_j$ konvex.
\end{Lemma}

\begin{Beispiel}
 $a_1,\dots,a_m,t_1,\dots,t_m\in\mathbb{R}, t_i\geq 0,i=1,\dots,m$.
 Wir definieren: 
 \begin{equation}
  f(x) := \sum\limits_{i=1}^mt_i|x-a_i|.
 \end{equation}
 Wir definieren weiter:
 \begin{equation}
  f_i(x):=|x-a_i|= \max\{x-a_i,-x+a_i\}.
 \end{equation}
Dann folgt, dass alle $f_i$ konvexe Funktionen sind und somit auch $f$.
\end{Beispiel}

\begin{Bemerkung}
 Offensichtlich ist das Minimum zweier konvexer Funktionen im Allgemeinen nicht konvex.
\end{Bemerkung}

\section{Das Subdifferential}
\begin{Definition}[Subdifferential]
 Sei $f:\mathbb{R}\rightarrow\mathbb{R}$ konvex. Dann heißt $s\in\mathbb{R}^n$ \textbf{Subgradient} von $f$ in $x$, 
wenn 
 \begin{equation}
  f(y) \geq f(x) + <s,y-x>, \forall y\in\mathbb{R}^n
 \end{equation}
gilt.
Das \textbf{Subdifferential} von $f$ in $x$ bezeichnet mit $\partial f(x)$, ist die Menge aller Subgradienten von $f$ in 
$x$.
\end{Definition}

\begin{Beispiel}[$f(x) = |x|$]
 Für $x=0$ und $s\in[-1,+1]$ und $y\in\mathbb{R}$ gilt
 \begin{equation}
  f(y) = |y| \geq sy = f(x)+<s,y-x> \Rightarrow [-1,+1]\in\partial f(0).
 \end{equation}
 Das Subdifferential von $f$ ist hierbei:
 \begin{equation}
  \partial f(x) = 
  \begin{cases}
   +1 & ,x > 0 \\
   [-1,+1] & ,x=0\\
   -1 & ,x < 0
  \end{cases}
 \end{equation}
\end{Beispiel}

\begin{Satz}
 Für eine konvexe Funktion $f:\mathbb{R}^n\rightarrow\mathbb{R}$ ist das Subdifferential $\partial f(x)$ nichtleer, 
konvex und kompakt.
\end{Satz}

\begin{Satz}
 Ist $f:\mathbb{R}^n\rightarrow\,\mathbb{R}$ konvex und differenzierbar, dann ist $\partial f(x) = \{\nabla f(x)\}$ 
einelementig.
\end{Satz}

\begin{Satz}
 Seien $f_1,f_2:\mathbb{R}^n\rightarrow\mathbb{R}$ konvex und überall endlich, und $t_1,t_2\in\mathbb{R};t_1,t_2\geq 
0$. Dann:
\begin{equation}
 \partial(t_1f_1+t_2f_2)(x) = t_1\partial f_1(x) + t_2\partial f_2(x), \forall x\in\mathbb{R}^n.
\end{equation}
\end{Satz}

\begin{Bemerkung}
 Dies ist erweiterbar auf 
 \begin{equation}
  \partial(\sum\limits_{i=1}^m t_if_i) = \sum_{i=1}^m f_i\partial f_i(x) 
 \end{equation}
Um also einen Subgradienten $s\in\partial f(x)$ zu erhalten, kann man beliebige Subgradienten $s^i\in\partial f_i(x)$ 
wählen und dann $s=\sum\limits_{i=1}^m t_is^i$ setzen.
\end{Bemerkung}

In vielen Anwendungen spielen Maximumfunktionen eine wichtige Rolle.

\begin{Satz}
 Mit konvexen Funktionen $f_i:\mathbb{R}^n\rightarrow\mathbb{R}, i=1,\dots,m$, sei die Funktion
 \begin{equation}
  f(x) = \max_{i=1,\dots,m} f_i(x), \forall x\in\mathbb{R}^n
 \end{equation}
 definiert.

 Für $x\in\mathbb{R}^n$ sei $I(x):=\{1\leq i\leq m|f(x)= f_i(x)\}$ die Menge der aktiven Indizes.
 Dann gilt für alle $x\in\mathbb{R}^n$:
 \begin{align}
  \partial f(x) & = \co\bigcup_{i\in I(x)}\partial f_i(x) \nonumber \\
  & = \left\{\sum\limits_{i\in I(x)} \alpha_i s^i | \alpha_i 
\geq 0, s^i \in \partial f_i(x), i\in I(x), \sum\limits_{i \in I(x)} \alpha_i = 1\right\}
 \end{align}
\end{Satz}

Um einen speziellen Subgradienten $s\in\partial f(x)$ zu berechnen, kann man also zunächst beliebige Subgradienten 
$s^i\in\partial f_i(x), i \in I(x)$, berechnen und dann eine Konvexkombination
\begin{align}
 s = \sum\limits_{i\in I(x)} \alpha_i s^i & & ,\alpha_i\geq 0, \sum\limits_{i\in I(x)} \alpha_i = 1
\end{align}
bilden. Insbesondere gilt: $s^i\in \partial f(x), \forall i \in I(x)$.

\begin{Korollar}
 Sind zusätzlich die Funktionen $f_i$, $i=1,\dots,m$ differenzierbar. dann gilt für alle $x\in\mathbb{R}^n$:
 \begin{equation}
  \partial f(x) = \co\{\nabla f_i(x)|i\in I(x)\}.
 \end{equation}
\end{Korollar}

\begin{Beispiel}[Maxq-Funktion]
 Sei $f:\R^n\rightarrow\R$, $f(x) = \max\limits_{1\leq i\leq n} x_i^2$ und $x = (x_1,\dots,x_n)\tr$
 Hier ist $f_i(x) = x_i^2, i=1,\dots,n$ und der Subgradient
 \begin{align}
  \partial f(x) = \co\{\nabla f_i(x)|i\in I(x)\} & & I(x) = \{1\leq i \leq n|f(x) = x_i^2\}
 \end{align}
Speziell gilt $\nabla f_i(x) \in \partial f(x), \forall i \in I(x)$.
\end{Beispiel}

\begin{Korollar}\label{kor:affin:nonsmooth}
 Zusätzlich seien die $f_i$ affine Funktionen, d.h $f_i(x) = <s^i,x> + r_i$ mit $s^i\in\R^n$, $r_i\in\R$, 
$i\in\{1,\dots,m\}\Rightarrow \partial f(x) = \co\{s^i|i\in I(x)\}$.
\end{Korollar}

\begin{Beispiel}[$\text{L}^\infty$-Norm]
 Sei $f_\infty(x_1,x_2)) = \max\limits_{1=1,\dots,m} |\xi_i x_1+x_2-\eta_i|$.
 
 Mit $g_i(x) = \xi_i x_1 +x_2 -\eta_i, i\in\{1,\dots,m\}$. $g_i$ kann dann auch geschrieben werden als $g_i(x) = 
<s,x>-\eta_i$, $s_i=(\xi_i,1)\tr$. Weiter sei $h_i(x) = |g_i(x)| = \max\{g_i(x),-g_i(x)\}$. $f_\infty$ kann nun 
geschrieben werden als:
\begin{equation}
 f_\infty(x) = \max\limits_{i=1,\dots,m} h_i(x)
\end{equation}
Laut Korollar \ref{kor:affin:nonsmooth} gilt:
\begin{equation}
 \partial  h_i(x) = 
 \begin{cases}
  \{s^i\} & ,g_i(x) > 0\\
  \{-s^i\} & ,g_i(x) < 0\\
  [-s^i,s^i] & ,g_i(x) = 0
 \end{cases}
\end{equation}

Wir definieren nun $\epsilon_i(x) \in \{1,-1\}$, genauer:
\begin{equation}
 \epsilon_i(x) = 
 \begin{cases}
  +1 & ,g_i(x) \geq 0\\
  -1 & ,g_i(x) < 0
 \end{cases}
\end{equation}
Daraus folgt $\epsilon_i(x) \cdot s^i \in\partial h_i(x).$ Mit $I(x) = \{1\leq i\leq m|f_\infty (x) = h_i(x)\}$ 
folgt $\co\{\epsilon_i(x)\cdot s^i|i\in I(x)\}\subseteq \partial f_\infty(x)$. Speziell gilt $\epsilon_i(x) \cdot 
s^i\in\partial f_\infty(x)$ für alle $i\in I(x)$.
\end{Beispiel}

\begin{Satz}
 Für eine konvexe Funktion $f:\R^n\rightarrow\R$ sind folgende Aussagen äquivalent:
 \begin{enumerate}
  \item[(1)] $x^*$ ist Lösung von $\min\limits_{x\in\R^n} f(x)$
  \item[(2)] $O_n\in\partial f(x^*)$.
 \end{enumerate}
\end{Satz}

\begin{proof}
 Es gilt $f(x)\geq f(x^*), \forall x\in\R^n$ genau dann, wenn (Definition Subgradient)
 \begin{equation}
  f(x) \geq f(x^*) + <0_n,x-x^*>, \forall x\in\R^n,
 \end{equation}
 was äquivalent zu $O_n\in\partial f(x^*)$ ist.
\end{proof}

\begin{Beispiel}[$f(x) = \norm{x}$]
 Es gilt $f(x) = 0 = f(0) + <O_n,x-0>\Rightarrow O_n\in\partial f(0_n)$. Daraus folgt $0_n$ ist ein Minimum von $f$.
\end{Beispiel}

\section{Das Subgradientenverfahren}
\begin{description}
 \item[Annahme:] Zu jedem Vektor $x\in\R^n$ kann man (mindestens) einen Subgradienten $s(x) \in\partial f(x)$ berechnen.
 \item[Verfahren:] Wähle einen Startpunkt $x^{(0)}\in\R^n, k=0$
 \begin{enumerate}
  \item Berechnen einen Subgradienten $s^{(k)}\in\partial f(x^{(k)}$
  \item Abbruchkriterium: Ist $s^{(k)} = 0_n$ (d.h. $0_n\in\partial f(x^{(k)})$), dann stoppe das Verfahren
  \item Setze 
  \begin{equation}
   d^{(k)} = -\frac{s^{(k)}}{\norm{s^{(k)}}}.
  \end{equation}
 Wähle eine Schrittweite $\sigma_k > 0 $  und setze 
 \begin{equation}
  x^{(k+1)} = x^{(k)}+\sigma_k\cdot d^{(k)}.
 \end{equation}
\item Setze $k=k+1$ und gehe zu $1.$
 \end{enumerate}
\end{description}

\begin{Bemerkung}
 Anstatt des Abbruchkriteriums $s^{(k)} = 0_n$ sollte in der Praxis 
 \begin{equation}
 \norm{x^{(k+1)}-x^{(k)}}\leq \epsilon 
 \end{equation}
 und eine maximale Iterationszahl genutzt werden. Weiter sollte die Abbruchbedingung 
 \begin{equation}
 \norm{f(x^{(k+1)})-f(x^{(k)})}\leq \epsilon 
 \end{equation}
 nicht genutzt werden. Die Schrittweite sollte nicht mit dem Armijo-Verfahren berechnet werden sondern mit
\begin{equation}
 \sigma_k = \frac{\sigma}{k+h}
\end{equation} 
oder
\begin{equation}
 \sigma_k = \frac{\sigma}{\sqrt k}
\end{equation} 
berechnet werden.
\end{Bemerkung}

\begin{Bemerkung}[Nachteile des Subgradientenverfahren]\
\begin{itemize}
  \item Die Suchrichtung $-s^{(k)}$ muss keine Abstiegsrichtung sein. 
  \begin{itemize}
   \item Die Schrittweise $\sigma_k$ kann nicht durch das Armijo-Verfahren bestimmt werden
   \item $\{f(x^{(k)}\}$ muss nicht monoton fallend sein 
  \end{itemize}
  $\Rightarrow$ kein Abstiegsverfahren
  \item Abbruchkriterium ist unrealistisch und praktisch nie erfüllt
 \end{itemize}
\end{Bemerkung}

\begin{Beispiel}[$f(x) = \norm{x}$]
 Das eindeutig bestimmte Minimum: $x^* = 0$.
 Starte das Verfahren in $x^{(0)} = x^* = 0$ und wähle $s^{(0)}\neq 0\Rightarrow$ Abbruchkriterium nicht erfüllt. D.h. 
im nächsten Schritt würde man sich sogar von der Lösung entfernen.
\end{Beispiel}

\begin{Lemma}
 Ist $f_\R^n\rightarrow\R$ konvex, $x^*$ ein beliebiges Minimum von $f$ und $x^{(k)}$ kein Minimum von $f$. Dann gibt 
es ein $T_k>0$, so dass für das Subgradientenverfahren gilt:
\begin{equation}
 \norm{x^{(k+1)}-x^*} < \norm{x^{(k)} - x^*}, \forall \sigma_k\in (0,T_k).
\end{equation}
\end{Lemma}

\begin{Bemerkung}[Geometrische Interpretation]
 Der Winkel zwischen der Richtung $-s^{(k)}$, in der wir uns ausgehend von $x^{(k)}$ bewegen und der Idealrichtung 
$x^*-x^{(k)}$ ist kleiner als $90^\circ$. D.h., bewegt man sich von $x^{(k)}$ aus Richtung $-s^{(k)}$ nicht zu weit, so 
ist $x^{(k+1)}$ näher an $x^*$ als 
$x^{(k)}$. Da man dieses $T_k$ praktisch nicht kennt, weiß man nur, dass die Schrittweite $\sigma_k$ hinreichend klein 
sind muss. 
Man verlangt daher
\begin{equation}
 \lim_{k\rightarrow\infty} \sigma_k = 0.
\end{equation}
Das allein reicht jedoch noch nicht aus. 
Ist nämlich $r:=\sum\limits_{k=0}^\infty \sigma_k < \infty$, dann gilt für alle $k$
\begin{equation}
 \norm{x^{(0)}-x^{(k)}} \leq \sum\limits_{i=0}^{k-1}\norm{x^{(i)}-x^{(i+1)}} = \sum\limits_{i=0}^{k-1} \sigma_i \leq r.
\end{equation}
D.h. alle $x^{(k)}$ liegen in der Kugel $\bar{B}(x^{(0)},r)$.
Um $x^*$ erreichen zu können muss $r$ hinreichend groß sein. Dies ist sichergestellt, wenn man 
\begin{equation}
 \sum\limits_{k=0}^\infty \sigma_k = \infty
\end{equation}
verlangt. Damit wird (aber nur) die Konvergenz der besten Funktionswerte garantiert.
\end{Bemerkung}

\begin{Bemerkung}
 Statt $\sum \sigma_k = \infty$ zu fordern kann man auch $\sum \sigma_k^2 < \infty$ fordern.
\end{Bemerkung}

\begin{Beispiel}
\begin{itemize}
 \item $\sigma_k = \frac{\sigma}{k+b}$ für festes $\sigma,b>0$\\(Spezialfall: $\sigma_k = \frac{1}{k})$
 \item $\sigma_k = \frac{\sigma}{\sqrt{k}}$ für festes $\sigma>0$\\(Spezialfall: $\sigma_k = \frac{1}{\sqrt k})$
\end{itemize}
\end{Beispiel}

\begin{Bemerkung}
 Im Gegensatz zum Gradientenverfahren, werden hier die Schrittweiten offline gesteuert.
 Ein Sinnvolles Abbruchkriterium ist demnach $\norm{x^{(k+1)}-x^{(k)}}\leq \epsilon_1$ ($\norm{s^{(k)} \leq 
\epsilon_2}$ oder die maximale Iterationszahl.
\end{Bemerkung}

\section{Laufzeitanalyse}
Sei $f\in\mathcal{F}^{0,0}_L$, d.h. $|f(x) - f(y)|\leq L\cdot\norm{x-y}$, konvex und Lipschitz stetig 
genau dann wenn $\norm{s}\leq L$ für alle $s\in\partial f(x), \forall x$.
\begin{align*}
 \norm{x^{(k+1)}-x^*}^2  & = \norm{x^{(k)} - \sigma_k\frac{s^{(k)}}{\norm{s^{(k)}}}-x^*}^2 \\
 & = 
\norm{x^{(k)}-x^*}-2\sigma_k\frac{s^{(k)}}{\norm{s^{(k)}}}\tr(x^{(k)}-x^*)+\sigma_k^2 \frac{{s^{(k)}}^2}{\norm
{s^{(k)}}^2}\\
& \leq \norm{x^{(k)}-x^*}^2-2\frac{\sigma_k}{\norm{s^{(k)}}}(f(x^{(k)})-f(x^*))+\sigma_k^2 \text{ (da Subgradient)}\\
& \overset{\norm{s^{(k)}}\leq L}{\leq}\norm{x^{(k)}-x^*}-2\frac{\sigma_k}{L}(f(x^{(k)})-f(x^*))+\sigma_k^2\\
\end{align*}
Das Aufsummieren von $i=0,\dots,k$ ergibt:
\begin{align*}
 0 &\leq \underbrace{\norm{x^{(k+1)}-x^*}^2}_{\geq 0}\\
 & \leq \norm{x^{(0)}-x^*}^2 - \frac{2}{L}\sum\limits_{i=0}^k\sigma_i(f(x^{(i)})-f(x^*))+\sum\limits_{i=0}^k \sigma_i^2
\end{align*}
Daraus folgt:
\begin{equation}
 \frac{2}{L}\sum\limits_{i=0}^k\sigma_i(f(x^{(i)})-f(x^*))\leq \norm{x^{(0)}-x^*}^2 +\sum\limits_{i=0}^k\sigma_i^2
\end{equation}
Es gilt:
\begin{align*}
 & & \left(\sum\limits_{i=0}^k\sigma_i\right)\min_i\{f(x^{(i)})-f(x^*)\} & \leq 
\sum\limits_{i=0}^k\sigma_i(f(x^{(i)})-f(x^*))\\
& \Rightarrow & 
\frac{2}{L}\left(\sum\limits_{i=0}^k\sigma_i\right)\min_i\{f(x^{(i)})-f(x^*)\} & \leq 
\norm{x^{(0)}-x^*}^2+\sum\limits_{i=0}^k\sigma_i^2 
\end{align*}
Dies wird nun umgeschrieben als:
\begin{equation}
 f_{best}^{(k)}-f(x^*) := \min_i f(x^{(i)})-f(x^*) \leq \frac{L}{2} 
\frac{\norm{x^{(0)}-x^*}^2 \sum\limits_{i=0}^k\sigma_i^2}{\sum\limits_{i=0}^ksigma_i}
\end{equation}
Die beste Strategie bei genau $k$ Iterationen ist:
\begin{equation}
 \sigma_i = \frac{\norm{x^{(0)} - x^*}}{\sqrt{k+1}}
\end{equation}
Woraus folgt:
\begin{equation}
 f_{best}^{(k)}-f(x^*) \leq \frac{L}{2}\frac{\norm{x^{(0)}-x^*}}{\sqrt{k+1}}
\end{equation}
Falls die Anzahl der Iterationen unbekannt ist wird $\sigma_i$ berechnet als:
\begin{equation}
 \sigma_i = \frac{h}{\sqrt{i+1}}
\end{equation}
Woraus folgt:
\begin{equation}
 f_{best}^{(k)} - f(x^*) \leq \frac{L}{2}\frac{\norm{x^{(0)}-x^*}^2\cdot h\cdot \log(k+1)}{h\cdot\sqrt{k+1}}
\end{equation}
Insgesamt kann nach $k$ Iterationen höchstens ein Fehler $\mathcal{O}\left(\frac{1}{\sqrt{k}}\right)$
erhalten werden. Soll stattdessen
\begin{equation}
 f_{best}^{(k)}-f(x^*) \leq \epsilon
\end{equation}
gelten, werden mindestens $\mathcal{O}\left(\frac{1}{\epsilon^2}\right)$ Iterationen benötigt ($\epsilon = 
\frac{1}{\sqrt{k}} \Leftrightarrow k = \frac{1}{\epsilon^2})$.










































\appendix
\chapter{Mathematische Grundlagen}

\begin{Definition}
  Eine symmetrische Matrix $A \in \R^{n \times n}$ hei\ss t \emph{positiv definit}, wenn
  \begin{gather*}
    x\tr A x > 0 \quad \forall x \in \R^n\,,\; x \neq 0_n
  \end{gather*}
  gilt. Die Matrix~$A$ hei\ss t \emph{positiv semidefinit}, wenn
  \begin{gather*}
    x\tr A x \geq 0 \quad \forall x \in \R^n
  \end{gather*}
  gilt.
\end{Definition}
\begin{Beispiel}
 $ A=I_2 = \begin{bmatrix}
		1 & 0\\ 0 & 1
 \end{bmatrix}
 $
 \begin{eqnarray*}
	x\tr A x & = &\begin{pmatrix}x_1\\x_2
	\end{pmatrix}\tr\begin{pmatrix}1 & 0\\ 0 & 1\end{pmatrix}\begin{pmatrix}x_1 \\ x_2\end{pmatrix}\\ & = &\begin{pmatrix}x_1 \\ x_2\end{pmatrix}\tr \begin{pmatrix}x_1 \\ x_2\end{pmatrix}\\ & = & x_1^2+x_2^2 > 0 ,
\end{eqnarray*}
$ \text{ wobei } \begin{pmatrix}x_1\\x_2\end{pmatrix} \neq \begin{pmatrix}0\\0\end{pmatrix}$.
\end{Beispiel}
%
\begin{Definition}[Differenzierbarkeit]
\label{def:diffbarkeit}
  Seien $f\colon D\to\R$ und $D\subseteq\R^n$ offen und existiere der Gradient
  \begin{equation*}
    \nabla f(x) = \left( \frac{\partial f}{\partial x_1}, \cdots, \frac{\partial f}{\partial x_n} \right)\tr
  \end{equation*}
  \noindent
  f\"ur alle $x \in D$.
  Dann hei\ss t $f$ \emph{differenzierbar} auf $D$.
\end{Definition}
%
\begin{Definition}[zweimalige Differenzierbarkeit]
\label{def:2diffability}
  Seien $f\colon D\to\R$ und $D\subseteq\R^n$ offen und existiere die Hessematrix $\nabla^2 f(x)\subseteq\mathcal{S}^n$ mit
  \begin{equation*}
    \nabla^2 f(x) = \left( \frac{\partial^2 f}{\partial x_i \partial x_j} \right)_{i, j \in [n]}
  \end{equation*}
  \noindent
  f\"ur alle $x \in D$.
  Dann hei"st $f$ \emph{zweimal differenzierbar} auf $D$.
\end{Definition}
\newglossaryentry{par:program}
{
  name=$\ldots$ program,
  description={Optimierungsproblem.
        Der Begriff \emph{programming} f\"ur \emph{mathematische Optimierung} r\"uhrt daher, dass Optimierungsmodule typischerweise die ersten Module sind, in denen Mathematikstudenten gezwungen sind, zu programmieren.
        Dies ist nat\"urlich nicht richtig.
        Tats\"achlich stammt der Begriff von der englischen \"Ubersetzung von Einsatzplan, bzw. Einsatzplanung.
        Die ersten spezifischen Optimierungsmethoden wurden im Milit\"ar entwickelt, weswegen sich insbesondere im englischen Sprachraum der Begriff \emph{programming} eingeb\"urgert hat.
        Ein \emph{program} ist daher ein \emph{Optimierungsproblem}.
        In j\"ungerer Zeit wird allerdings \emph{optimisation}, respektive \emph{optimization} immer st\"arker verwendet}
}

\newglossaryentry{par:conv}
{
  name=convex,
  description={konvex}
}

\newglossaryentry{par:set}
{
  name=set,
  description={Menge}
}

\newglossaryentry{par:func}
{
  name=function,
  description={Funktion, Abbildung}
}

\newglossaryentry{par:map}
{
  name={mapping, map},
  description={Abbildung, Funktion}
}

\newglossaryentry{par:cone}
{
  name=cone,
  description={Kegel. \textsl{conic} -- kegelförmig.}
}

\newglossaryentry{par:hull}
{
  name=hull,
  description={H\"ulle, H\"ullenoperator.}
}

\newglossaryentry{par:conti}
{
  name=continuous,
  description={stetig. \emph{continuity} -- Stetigkeit.}
}

\glsaddall
\printglossary[title=Optimisational English]


\end{document}